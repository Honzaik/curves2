\documentclass[12pt, a4paper]{article}
\usepackage[margin=1in]{geometry}
\usepackage[utf8x]{inputenc}
\usepackage{indentfirst} %indentace prvního odstavce
\usepackage{mathtools}
\usepackage{amsfonts}
\usepackage{amsmath}
\usepackage{amssymb}
\usepackage{graphicx}
\usepackage{enumitem}
\usepackage{subfig}
\usepackage{float}
\usepackage[czech]{babel}
\usepackage{mathdots}
\usepackage{slashbox}

\begin{document}
\begin{center}
\large NMMB538 - DÚ3

\normalsize Jan Oupický
\end{center}
\vspace{1\baselineskip}

\section{}
\begin{enumerate}[label=(\alph*)]
    \item $x \in P \subset O_P$. Z definice $O_P$ víme, že $K \subset O_P \implies K[x] \subset O_P$. Zřejmě $x\notin O_P$, protože jinak by $O_P = F$. Označme $I \coloneqq P \cap K[x]$. $I$ je prvoideál v $K[x]$, tedy je tvaru $I = (f), f \in K[x], f$ ireducibilní. Označme $R \coloneqq K[x]_{(f)} = \{ \frac{a}{b} | a \in K[x], b \in K[x] \setminus (f) \}$. 

    $R \subseteq O_P$, protoze $O_P = \{a \in F | v_P(a) \geq 0 \}, P = \{ a \in F | v_P(a) \geq 1 \} \implies \frac{a}{b} \in R: v_P(\frac{a}{b}) = v_P(a)-v_P(b)$ z definice $v_P(b)=0$, protože $b \notin (f) \subset P$ a $a \in K[x] \in O_P \implies v_P(a) \geq 0 \implies v_P(\frac{a}{b} \geq 0$. 

    Zároveň je $R$ také valuační okruh $F$. Protože $\frac{a}{b} \in F \iff a \in K[x], b \in K[x]\setminus 0$. Buď $b \notin (f) \implies \frac{a}{b} \in R, a \notin (f), b \in (f) \implies \frac{b}{a} \in R$ a nebo $a,b \in (f)$ a to se dá vydělit na jeden z přechozích případů. Nechť $Q$ je daný jediný maximální ideál $R$. Máme tedy $Q \subset R \subseteq O_P$. Z maximality $P$ tedy plyne, že $Q = P$ a tedy musí platit $R = O_P$.

    \item $\Rightarrow:$ $P' \subset P \implies a\in F : v_{P'}(a) = e(P'|P)\cdot v_P(a)$ kde $e(P'|P)\geq 1$. $x \in P$ z definice $P$, tedy $v_P(x) > 0 \implies v_{P'} > 0$ z předchozí rovnosti.

    $\Leftarrow:$ Označme $Q \coloneqq P' \cap F$. $v_{P'}(x)\geq 0, x \in F \implies v_Q(x) \geq 0$. Tedy $Q$ je místo $K(x)$ obsahující $x$. Víme, že existuje jediné takové místo $F/K$, protože \\$1=[F:K(x)]\geq \sum_{P:x\in P} v_P(x)\deg(P)$ . Takže $P' \subset Q = P \implies P' | P$.

    \item Z předchozího bodu víme, že $v_P(x)=1$ a $\deg_{F/K}(P)=1$. Tudíž $e(P'|P)=v_{P'}(x)$. Stejně tak dle prop F.6, kde $K'=K, \deg_{F/K}(P)=1 \implies f(P'|P)=\deg_{F'/K}(P')$.
    \item Označíme-li $n = [F':F]$, rozšíření je konečné, jelikož $F'$ je algebraické funkční těleso nad $K$ a $F=K(x)$ a $x$ je transcendentní nad $K$.

    Použijeme-li značení a předpoklady věty F.7 pro naše $P$ obsahující $x$ a předchozí bod. Dostaneme tedy \\$[F':F]=\sum_i v_{P_i}(x)\cdot \deg_{F'/K}(P_i)$
\end{enumerate}

\section{}
Označme $w(x,y)=y^2-x^3-ax-b$.
\begin{enumerate}[label=(\alph*)]
    \item Z předchozího úkolu víme, že pokud $w$ je smooth, tak $F/K(x)$ je separabilní. Také víme, že $F/K(x)$ je konečné. Dále $F$ je jednoduché rozšíření jelikož $F = \{\frac{a+(w)}{b+(w)}| a \in K[x,y], b \in K[x,y] \setminus 0 \} \supset \{\frac{a+(w)}{b+(w)}| a \in K[x], b \in K[x] \setminus 0 \} \cong K(x)$. Tedy lehce nepřesně můžeme napsat, že $K(x)=K(x+(w)) \implies F = K(x+(w))(y+(w))$. Budeme ale používat zjednodušené značení, jako v předchozím úkolu. 

    Tedy $[F:K(x)]=2$, $m_{y,K(x)}(T)=T^2 - x^3 - ax -b$. Kořeny tohoto polynomu jsou $y,-y \in F$ . Víme, že $y$ je separabilní nad $K(x)$ tedy  $|\text{Hom}(F,\bar{K(x)})|=[F:K(x)]=2$. Oba tyto homomorfismy permutují kořeny $m_{y,K(x)}$ a oba tyto kořeny jsou v $F$. Takze je $F/K(x)$ normální a Galoisovo.
    \item Pokud $t=y+\lambda x + \mu$, protíná $C=V_w$ právě ve 2 různých bodech, tak pro dané $(x,y)$ platí $y=-\lambda x - \mu, w(x,y)=0 \implies w(x,-\lambda x - \mu) = 0$, kde $g(x)=w(x,-\lambda x - \mu) = -x^3+x^2 \lambda^2 + x(2\lambda \mu -a)+\mu^2 -b \in K[x]$. Tento polynom je stupně $3$ a dle zadání má jen 2 různé kořeny, takže z definice není separabilní. 

    Pokud najdeme prvek $a \in F$, tž. $g$ je minimální polynom $a$ nad $K(t)$, tak $F$ není separabilní a tudíž ani Galoisovo.
\end{enumerate}

\end{document}

