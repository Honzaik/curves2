\documentclass[12pt, a4paper]{article}
\usepackage[margin=1in]{geometry}
\usepackage[utf8x]{inputenc}
\usepackage{indentfirst} %indentace prvního odstavce
\usepackage{mathtools}
\usepackage{amsfonts}
\usepackage{amsmath}
\usepackage{amssymb}
\usepackage{graphicx}
\usepackage{enumitem}
\usepackage{subfig}
\usepackage{float}
\usepackage[czech]{babel}
\usepackage{mathdots}
\usepackage{slashbox}

\begin{document}
\begin{center}
\large NMMB538 - DÚ3

\normalsize Jan Oupický
\end{center}
\vspace{1\baselineskip}

\section{}
\begin{enumerate}[label=(\alph*)]
    \item $x \in P \subset O_P$. Z definice $O_P$ víme, že $K \subset O_P \implies K[x] \subset O_P$. Zřejmě $x\notin O_P$, protože jinak by $O_P = F$. Označme $I \coloneqq P \cap K[x]$. $I$ je prvoideál v $K[x]$, tedy je tvaru $I = (f), f \in K[x], f$ ireducibilní. Označme $R \coloneqq K[x]_{(f)} = \{ \frac{a}{b} | a \in K[x], b \in K[x] \setminus (f) \}$. 

    $R \subseteq O_P$, protoze $O_P = \{a \in F | v_P(a) \geq 0 \}, P = \{ a \in F | v_P(a) \geq 1 \} \implies \frac{a}{b} \in R: v_P(\frac{a}{b}) = v_P(a)-v_P(b)$ z definice $v_P(b)=0$, protože $b \notin (f) \subset P$ a $a \in K[x] \in O_P \implies v_P(a) \geq 0 \implies v_P(\frac{a}{b} \geq 0$. 

    Zároveň je $R$ také valuační okruh $F$. Protože $\frac{a}{b} \in F \iff a \in K[x], b \in K[x]\setminus 0$. Buď $b \notin (f) \implies \frac{a}{b} \in R, a \notin (f), b \in (f) \implies \frac{b}{a} \in R$ a nebo $a,b \in (f)$ a to se dá vydělit na jeden z přechozích případů. Nechť $Q$ je daný jediný maximální ideál $R$. Máme tedy $Q \subset R \subseteq O_P$. Z maximality $P$ tedy plyne, že $Q = P$ a tedy musí platit $R = O_P$.

    \item $\Rightarrow:$ $P' \subset P \implies a\in F : v_{P'}(a) = e(P'|P)\cdot v_P(a)$ kde $e(P'|P)\geq 1$. $x \in P$ z definice $P$, tedy $v_P(x) > 0 \implies v_{P'} > 0$ z předchozí rovnosti.

    $\Leftarrow:$ Označme $Q \coloneqq P' \cap F$. $v_{P'}(x)\geq 0, x \in F \implies v_Q(x) \geq 0$. Tedy $Q$ je místo $K(x)$ obsahující $x$. Víme, že existuje jediné takové místo $F/K$, protože \\$1=[F:K(x)]\geq \sum_{P:x\in P} v_P(x)\deg(P)$ . Takže $P' \subset Q = P \implies P' | P$.

    \item Z předchozího bodu víme, že $v_P(x)=1$ a $\deg_{F/K}(P)=1$. Tudíž $e(P'|P)=v_{P'}(x)$. Stejně tak dle prop F.6, kde $K'=K, \deg_{F/K}(P)=1 \implies f(P'|P)=\deg_{F'/K}(P')$.
    \item Označíme-li $n = [F':F]$, rozšíření je konečné, jelikož $F'$ je algebraické funkční těleso nad $K$ a $F=K(x)$ a $x$ je transcendentní nad $K$.

    Použijeme-li značení a předpoklady věty F.7 pro naše $P$ obsahující $x$ a předchozí bod. Dostaneme tedy \\$[F':F]=\sum_i v_{P_i}(x)\cdot \deg_{F'/K}(P_i)$
\end{enumerate}

\section{}
Označme $w(x,y)=y^2-x^3-ax-b$.
\begin{enumerate}[label=(\alph*)]
    \item Z předchozího úkolu víme, že pokud $w$ je smooth, tak $F/K(x)$ je separabilní. Také víme, že $F/K(x)$ je konečné. Dále $F$ je jednoduché rozšíření jelikož $F = \{\frac{a+(w)}{b+(w)}| a \in K[x,y], b \in K[x,y] \setminus 0 \} \supset \{\frac{a+(w)}{b+(w)}| a \in K[x], b \in K[x] \setminus 0 \} \cong K(x)$. Tedy lehce nepřesně můžeme napsat, že $K(x)=K(x+(w)) \implies F = K(x+(w))(y+(w))$. Budeme ale používat zjednodušené značení, jako v předchozím úkolu. 

    Tedy $[F:K(x)]=2$, $m_{y,K(x)}(T)=T^2 - x^3 - ax -b$. Kořeny tohoto polynomu jsou $y,-y \in F$ . Víme, že $y$ je separabilní nad $K(x)$ tedy  $|\text{Hom}(F,\bar{K(x)})|=[F:K(x)]=2$. Oba tyto homomorfismy permutují kořeny $m_{y,K(x)}$ a oba tyto kořeny jsou v $F$. Takze je $F/K(x)$ normální a Galoisovo.
    \item Pokud $t=y+\lambda x + \mu$, protíná $C=V_w$ právě ve 2 různých bodech, tak pro dané $(x,y)$ platí $y=-\lambda x - \mu, w(x,y)=0 \implies w(x,-\lambda x - \mu) = 0$, kde $g(x)=w(x,-\lambda x - \mu) = -x^3+x^2 \lambda^2 + x(2\lambda \mu -a)+\mu^2 -b \in K[x]$. Tento polynom je stupně $3$ a dle zadání má jen 2 různé kořeny tedy má násobný kořen $\implies g(x)=-(x-c_1)^2(x-c_2) \in K[x]$. 

    Dále nevím.

    \item Zjistíme, kdy je $F/K(y)$ normální. $m_{x,K(y)}(T) = -T^3-aT-b-y^2 \in K(y)[T]$. Víme, že kořen v $F$ je $x$, polynom tedy vydělíme v F $\frac{m_{x,K(y)}(T)}{T-x} = -T^2-Tx-x^2-a$. Z toho nám vyjde, že další kořeny $m_{x,K(y)}(T)$ tedy jsou $x_{1,2} = \frac{1}{2}(-x\pm \sqrt{-3x^2-4a})$. Pokud $x_{1,2} \in F$ tak je $F/K(y)$ Galoisovo. 

    Zajímá nás tedy kdy $x_{1,2} \notin F$. Speciálně $\sqrt{-3x^2-4a} \notin F$. Pokud tedy například $a = 0$, tak $\sqrt{-3x^2-4a}=\sqrt{-3}x = \sqrt{2}x$ když $K = \mathbb{Z}_5$. V $\mathbb{Z}_5$ neexistuje $\sqrt{2}$ tedy kořen není v $F$, takže pokud $a=0$, tak $F/K(y)$ není normální tedy ani Galoisovo.
\end{enumerate}

\section{}
Označme $w(x,y)=y^2-x^3-ax^2-bx \in K[x,y], a^2-4b \neq 0, b \neq 0$.
\begin{enumerate}[label=(\alph*)]
    \item Takové $z$ nemůže být algebraické nad $K$. Jelikož $[F':K]= \infty $. Předpokládejme tedy, že existuje $z \in F': F'=K(z), z$ transcendentní nad $K$, tedy $K(z) \cong K(x)$. Víme ale, že $[F':K(x)]=2$, tedy $F' \neq K(x) \cong K(z)$.

    \item $s = \frac{b-x^2}{x} = \frac{bx-x^3}{x^2}$ použijme rovnost v $F'$ $x^3=y^2-ax^2-bx \implies$
    \[s = \frac{bx-(y^2-ax^2-bx)}{x^2} = \frac{-y^2+ax^2+2bx}{x^2}=-\frac{y^2}{x^2}+a+\frac{2b}{x} \implies
    \]
    \[
    s = -t^2+a+\frac{2b}{x} \implies s+t^2-a = \frac{2b}{x} \iff x = \frac{2b}{s+t^2-a}
    \]
    \[
    y = t\cdot \frac{2b}{s+t^2-a} = \frac{y}{x} x
    \]
    Umíme vyjádřit $x,y$ pomocí $s,t$ tedy $K(s,t)=K(x,y)$ kde $w(x,y)=0$
    \item Porovnáme strany a využijeme rovnosti $y^2 =x^3+ax^2+bx$, která platí v $F'$:
    \[
    s^2 = \left(\frac{b-x^2}{x}\right)^2 = \frac{b^2-2bx^2+x^4}{x^2} = x^2-2b+\frac{b^2}{x^2}
    \]
    \begin{gather*}
    t^4-2at^2+a^2-4b=\frac{y^4}{x^4}-2a\frac{y^2}{x^2} + a^2-4b = \frac{y^4-2ax^2y^2+a^2x^4-4bx^4}{x^4} \implies\\
    \frac{(x^3+ax^2+bx)^2-2ax^2(x^3+ax^2+bx)+a^2x^2-4bx^4}{x^4} = \frac{x^6-2bx^4+b^2x^2}{x^4} = s^2
    \end{gather*}

    \item $K(t^2,st)=K(t^2)(st)$. $st \notin K(t^2)$ protože $st=\frac{yb-x^2y}{x^2}$ a monočlen $x^2y$ nemůžeme dostat jako prvek $K(\frac{y^2}{x^2})$. Zároveň $m_{st,K(t^2)}(T) = T^2-(st)^2$. Je to opravdu min. poly nad $K(t)$ protože výše jsme ukázali, že $s^2 \in K(t^2)\implies (st)^2=s^2t^2 \in K(t^2)$.

    Tedy $z \in K(t^2,st): z = \frac{f+g\cdot st}{h+j\cdot st}, f,g,h,j \in K(t^2) (\iff f = f'(t^2), \dots, f' \in K[x]) \implies z = \frac{f'(t^2)+g'(t^2)st}{h'(t^2)+j'(t^2)st} \cdot \frac{h'(t^2)-j'(t^2)st}{h'(t^2)-j'(t^2)st} = \frac{a(t^2)+b(t^2)st}{c(t^2)}$

    Pokud tedy existují $f,g,h \in K[x]: t = \frac{f(t^2)+s\cdot g(t^2)}{h(t^2)} \implies$
    \[th\left(\frac{y^2}{x^2}\right)=\frac{y}{x}h\left(\frac{y^2}{x^2}\right) = f\left(\frac{y^2}{x^2}\right)+\frac{b-x^2}{x}g\left(\frac{y^2}{x^2}\right)\]
    Na levé strance se $y$ vyskytuje v liché mocnině, ale na pravé vždy v sudé, tudíž rovnost nemůže platit.
    \item Víme, že $t \notin K(t^2,st)$. Zároveň ale $F'=K(s,t)=K(t^2,st)(t)$, protože $s=st\cdot t^{-1}$. $m_{t,K(t^2,st)}(T)=T^2-t^2$, tento polynom má za kořeny $t,-t$ a je ireducibilní v $K(t^2,st)$ protože $t,-t \notin K(t^2,st)$. Tedy $F'/K(t^2,st)$ je tedy jednoduché algebraické rozšíření konečného stupně $[F':K(t^2,st)]=\deg m_{t,K(t^2,st)} = 2$.
\end{enumerate}

\end{document}

