\documentclass[12pt, a4paper]{article}
\usepackage[margin=1in]{geometry}
\usepackage[utf8x]{inputenc}
\usepackage{indentfirst} %indentace prvního odstavce
\usepackage{mathtools}
\usepackage{amsfonts}
\usepackage{amsmath}
\usepackage{amssymb}
\usepackage{graphicx}
\usepackage{enumitem}
\usepackage{subfig}
\usepackage{float}
\usepackage[czech]{babel}
\usepackage{mathdots}
\usepackage{slashbox}

\begin{document}
\begin{center}
\large NMMB538 - DÚ3

\normalsize Jan Oupický
\end{center}
\vspace{1\baselineskip}

\section{}
\begin{enumerate}[label=(\alph*)]
    \item $x \in P \subset O_P$. Z definice $O_P$ víme, že $K \subset O_P \implies K[x] \subset O_P$. Zřejmě $x\notin O_P$, protože jinak by $O_P = F$. Označme $I \coloneqq P \cap K[x]$. $I$ je prvoideál v $K[x]$, tedy je tvaru $I = (f), f \in K[x], f$ ireducibilní. Označme $R \coloneqq K[x]_{(f)} = \{ \frac{a}{b} | a \in K[x], b \in K[x] \setminus (f) \}$. 

    $R \subseteq O_P$, protoze $O_P = \{a \in F | v_P(a) \geq 0 \}, P = \{ a \in F | v_P(a) \geq 1 \} \implies \frac{a}{b} \in R: v_P(\frac{a}{b}) = v_P(a)-v_P(b)$ z definice $v_P(b)=0$, protože $b \notin (f) \subset P$ a $a \in K[x] \in O_P \implies v_P(a) \geq 0 \implies v_P(\frac{a}{b}) \geq 0$. 

    Zároveň je $R$ také valuační okruh $F$. Protože $\frac{a}{b} \in F \iff a \in K[x], b \in K[x]\setminus 0$. Buď $b \notin (f) \implies \frac{a}{b} \in R, a \notin (f), b \in (f) \implies \frac{b}{a} \in R$ a nebo $a,b \in (f)$ a to se dá vydělit na jeden z přechozích případů. Nechť $Q$ je daný jediný maximální ideál $R$. Máme tedy $Q \subset R \subseteq O_P$. Z maximality $P$ tedy plyne, že $Q = P$ a tedy musí platit $R = O_P$.

    Díky charakterizaci valuací na $K(x)$ víme, že místo, které obsahuje $P$ je definováno valuací $v_x$, pro kterou platí $v_x(a/b) = mult(a)-mult(b), \frac{a}{b} \in K(x)$. Pokud tedy použijeme definici $O_P = \{a \in F: v_x(a) \ge 0\}$. Pak $O_P$ můžeme definovat alternativně jako $O_P = \{ \frac{a}{b} | a,b \in K[x], b\neq 0, mult(a) \ge mult(b)\}$

    \item $\Rightarrow:$ $P' \subset P \implies a\in F : v_{P'}(a) = e(P'|P)\cdot v_P(a)$ kde $e(P'|P)\geq 1$. $x \in P$ z definice $P$, tedy $v_P(x) > 0 \implies v_{P'}(a) > 0$ z předchozí rovnosti.

    $\Leftarrow:$ Označme $Q \coloneqq P' \cap F$. $v_{P'}(x)\geq 0, x \in F \implies v_Q(x) \geq 0$. Tedy $Q$ je místo $K(x)$ obsahující $x$. Víme, že existuje jediné takové místo $F/K$, protože \\$1=[F:K(x)]\geq \sum_{P:x\in P} v_P(x)\deg(P)$ . Takže $P' \supset Q = P \implies P' | P$.

    \item Z předchozího bodu víme, že $v_P(x)=1$ a $\deg_{F/K}(P)=1$. Tudíž $e(P'|P)=v_{P'}(x)$. Stejně tak dle prop F.6, kde $K'=K, \deg_{F/K}(P)=1 \implies f(P'|P)=\deg_{F'/K}(P')$.
    \item Označíme-li $n = [F':F]$, rozšíření je konečné, jelikož $F'$ je algebraické funkční těleso nad $K$ a $F=K(x)$ a $x$ je transcendentní nad $K$.

    Použijeme-li značení a předpoklady věty F.7 pro naše $P$ obsahující $x$ a předchozí bod. Dostaneme tedy \\$[F':F]=\sum_i v_{P_i}(x)\cdot \deg_{F'/K}(P_i)$.

    Všimneme si, že prvek $(x)_+ \in \text{Div}(F'/K)$, který je definován jako\\ $(x)_+ = \sum\limits_{P \in \mathbb{P}_{F'/K}:x \in P} v_P(x)\deg(P)$, odpovídá $[F':F]$. 
\end{enumerate}

\section{}
Označme $w(x,y)=y^2-x^3-ax-b$.
\begin{enumerate}[label=(\alph*)]
    \item Z předchozího úkolu víme, že pokud $w$ je smooth, tak $F/K(x)$ je separabilní. Také víme, že $F/K(x)$ je konečné. Dále $F$ je jednoduché rozšíření jelikož $F = \{\frac{a+(w)}{b+(w)}| a \in K[x,y], b \in K[x,y] \setminus 0 \} \supset \{\frac{a+(w)}{b+(w)}| a \in K[x], b \in K[x] \setminus 0 \} \cong K(x)$. Tedy lehce nepřesně můžeme napsat, že $K(x)=K(x+(w)) \implies F = K(x+(w))(y+(w))$. Budeme ale používat zjednodušené značení, jako v předchozím úkolu. 

    Tedy $[F:K(x)]=2$, $m_{y,K(x)}(T)=T^2 - x^3 - ax -b$. Kořeny tohoto polynomu jsou $y,-y \in F$ . Víme, že $y$ je separabilní nad $K(x)$ tedy  $|\text{Hom}_K(F,\bar{K}(x))|=[F:K(x)]=2$. Oba tyto homomorfismy permutují kořeny $m_{y,K(x)}$ a oba tyto kořeny jsou v $F$. Takže je $F/K(x)$ normální a tedy Galoisovo.

    \item Z minulého semestru víme, že pokud $t=y+\lambda x + \mu$, $\gamma \in V_w(K) \cap V_t(K)$ a $|V_w \cap V_t| > 1$, tak existují body $\delta_1, \delta_2 \in V_w(K): V_w \cap V_t =\{\gamma, \delta_1, \delta_2\}$ a $t$ je tečnou $w$ v bodě $\gamma$ právě když pokud $\gamma \in \{\delta_1,\delta_2\}$. V našem případě jsou průsečíky pouze 2, tedy $\delta \coloneqq \delta_1 = \delta_2$. A máme $V_w \cap V_t =\{\gamma, \delta\}$ a $t$ je tečnou pouze v jednom bodě.

    Označme místa příslušná těmto průsečíkům $P_1', P_2' \in \mathbb{P}_{F/K}$. Nechť $P_1'$ je místo příslušné průsečíku, kde je $t$ tečnou. Poté platí $v_{P'_1}(t) \geq 2$ a $v_{P'_2}(t) = 1$. 

    Označme nyní $P_1 = P'_1 \cap K(t), P_2 = P'_2 \cap K(t)$. $P_1, P_2$ jsou prvky $\mathbb{P}_{K(t)/K}$. Ze stejného důvodu, proč existuje jediné místo $K(x)/K$ obsahující $x$ existuje jediné místo $P \in \mathbb{P}_{K(t)/K}$ obsahující $t$. $P_1,P_2$ zřejmě z definice obsahují $t$ jelikož $P'_1, P'_2$ obsahují $t$. Tedy $P \coloneqq P_1 = P_2 \implies P'_1 | P$ a  $P'_2 | P$.

    Nyní předpokládejme pro spor, že $F/K(t)$ je Galoisovo. Dle proposition F.15 existuje $\sigma \in \text{Gal}(F|K(t))$ takové, že $\sigma(P'_1)=P'_2$. Z definice $\sigma$ platí $\sigma^{-1}(t)=t$, tedy dle F.9 platí $v_{\sigma(P'_1)}(t)=v_{P'_2}(t) = v_{P'_1}(\sigma^{-1}(t))=v_{P'_1}(t)$ což je spor s valuacemi spočtenými výše. Tedy $F/K(t)$ není Galoisovo.

    \item Zjistíme, kdy je $F/K(y)$ normální. $m_{x,K(y)}(T) = -T^3-aT-b-y^2 \in K(y)[T]$. Víme, že kořen v $F$ je $x$, polynom tedy vydělíme v F $\frac{m_{x,K(y)}(T)}{T-x} = -T^2-Tx-x^2-a$. Z toho nám vyjde, že další kořeny $m_{x,K(y)}(T)$ tedy jsou $x_{1,2} = \frac{1}{2}(-x\pm \sqrt{-3x^2-4a})$. Pokud $x_{1,2} \in F$ tak je $F/K(y)$ Galoisovo. 

    Zajímá nás tedy kdy $x_{1,2} \notin F$. Speciálně $\sqrt{-3x^2-4a} \notin F$. Pokud tedy například $a = 0$, tak $\sqrt{-3x^2-4a}=\sqrt{-3}x = \sqrt{2}x$ když $K = \mathbb{Z}_5$. V $\mathbb{Z}_5$ neexistuje $\sqrt{2}$ tedy kořen není v $F$, takže pokud $a=0$, tak $F/K(y)$ není normální tedy ani Galoisovo.
\end{enumerate}

\section{}
Označme $w(x,y)=y^2-x^3-ax^2-bx \in K[x,y], a^2-4b \neq 0, b \neq 0$.
\begin{enumerate}[label=(\alph*)]
    \item Víme, že $F'$ lze vyjádřit jako $K(z)$ pro nějaké $z \in F'$ právě když genus $F'$ je roven 0. Zároveň $F'$ je eliptické funkční těleso právě když $w$ je hladké. Eliptické funkční těleso má genus 1. Chceme tedy dokázat, že $w$ je hladké.
    \begin{gather*}
    \frac{\partial w}{\partial x}(x,y) = -3x^2-2ax-b\\
    \frac{\partial w}{\partial y}(x,y) = 2y\\
    \end{gather*}
    Spočteme kořeny polynomu $\frac{\partial w}{\partial x}(x,y)$, které jsou $x_{1,2} = \frac{-a \pm \sqrt{a^2-3b}}{3}$. Kanditáti na singularitu jsou tedy body $(x_1,0), (x_2,0)$. Ověříme, zda leží na křivce neboli $w(x_1,0)=x_1(-x_1^2-ax_1-b)=0$ nebo $w(x_2,0)=0$. Spočteme to pro případ $x_1$, pro $x_2$ to jde obdobně.

    Kdy te jedy $x_1 = 0$ nebo $-x_1^2-ax_1-b=0$
    \begin{gather*}
    x_1 = 0 \iff \frac{-a + \sqrt{a^2-3b}}{3} = 0 \iff \sqrt{a^2-3b} = a \iff b = 0\\
    -x_1^2-ax_1-b=0 \iff a^2 + a\sqrt{a^2 - 3 b}- 6 b = 0 \iff \\
    a^2 - 6 b = a\sqrt{a^2 - 3 b} \iff a^4 - 12 a^2 b + 36 b^2 = a^4 - 3 a^2 b \iff \\
    36 b^2 = 8 a^2 b \iff b = \frac{a^2}{4}
    \end{gather*}
    Což dle předpokládů nejde. Tedy je $w$ hladké a tedy i $F$ má genus 1 tedy to není jednoduché rozšíření.

    \item $s = \frac{b-x^2}{x} = \frac{bx-x^3}{x^2}$ použijme rovnost v $F'$ $x^3=y^2-ax^2-bx \implies$
    \[s = \frac{bx-(y^2-ax^2-bx)}{x^2} = \frac{-y^2+ax^2+2bx}{x^2}=-\frac{y^2}{x^2}+a+\frac{2b}{x} \implies
    \]
    \[
    s = -t^2+a+\frac{2b}{x} \implies s+t^2-a = \frac{2b}{x} \iff x = \frac{2b}{s+t^2-a}
    \]
    \[
    y = t\cdot \frac{2b}{s+t^2-a} = \frac{y}{x} x
    \]
    Umíme vyjádřit $x,y$ pomocí $s,t$ tedy $K(s,t)=K(x,y)$ kde $w(x,y)=0$
    \item Porovnáme strany a využijeme rovnosti $y^2 =x^3+ax^2+bx$, která platí v $F'$:
    \[
    s^2 = \left(\frac{b-x^2}{x}\right)^2 = \frac{b^2-2bx^2+x^4}{x^2} = x^2-2b+\frac{b^2}{x^2}
    \]
    \begin{gather*}
    t^4-2at^2+a^2-4b=\frac{y^4}{x^4}-2a\frac{y^2}{x^2} + a^2-4b = \frac{y^4-2ax^2y^2+a^2x^4-4bx^4}{x^4} \implies\\
    \frac{(x^3+ax^2+bx)^2-2ax^2(x^3+ax^2+bx)+a^2x^2-4bx^4}{x^4} = \frac{x^6-2bx^4+b^2x^2}{x^4} = s^2
    \end{gather*}

    \item $K(t^2,st)=K(t^2)(st)$. $st \notin K(t^2)$ protože $st=\frac{yb-x^2y}{x^2}$ a monočlen $x^2y$ nemůžeme dostat jako prvek $K(\frac{y^2}{x^2})$. Zároveň $m_{st,K(t^2)}(T) = T^2-(st)^2$. Je to opravdu min. poly nad $K(t)$ protože výše jsme ukázali, že $s^2 \in K(t^2)\implies (st)^2=s^2t^2 \in K(t^2)$.

    Tedy $z \in K(t^2,st): z = f+g\cdot st, f,g \in K(t^2)$ neboli existují $u,v,w,z \in K[x]: f = \frac{u(t^2)}{v(t^2)}, g = \frac{w(t^2)}{z(t^2)} \implies z = \frac{u(t^2)}{v(t^2)} + \frac{w(t^2)\cdot st}{z(t^2)}$. Převedeme na společný jmenovatel: $z = \frac{u\cdot z(t^2) + v\cdot w(t^2)\cdot st}{v\cdot z (t^2)}$. Tedy máme požadovaný tvar, jelikož $u\cdot z \in K[x]$ a stejně tak zbylé 2 polynomy. 

    Pokud tedy existují $f,g,h \in K[x]: t = \frac{f(t^2)+st\cdot g(t^2)}{h(t^2)} \implies$
    \[t \cdot h\left(\frac{y^2}{x^2}\right)=\frac{y}{x} \cdot h\left(\frac{y^2}{x^2}\right) = f\left(\frac{y^2}{x^2}\right)+\frac{yb-yx^2}{x^2} \cdot g\left(\frac{y^2}{x^2}\right) =\]
    \[f\left(\frac{y^2}{x^2}\right)+\frac{yb}{x^2} \cdot g\left(\frac{y^2}{x^2}\right) - y\cdot g\left(\frac{y^2}{x^2}\right)\]
    Na levé strance se $y$ vyskytuje v liché mocnině, ale na pravé v liché i sudé, takže musí $f = 0$. Poté máme
    \[
    \frac{y}{x} \cdot h\left(\frac{y^2}{x^2}\right)=\frac{yb}{x^2} \cdot g\left(\frac{y^2}{x^2}\right) - y\cdot g\left(\frac{y^2}{x^2}\right)
    \]
    Obdobně můžeme argumentovat, že na levé straně se nenachází zlomky tvaru $\frac{y^i}{x^{j}}: i \neq j$. $t$ tedy nejde takto vyjádřit.

    \item Víme, že $t \notin K(t^2,st)$. Zároveň ale $F'=K(s,t)=K(t^2,st)(t)$, protože $s=st\cdot t^{-1}$. $m_{t,K(t^2,st)}(T)=T^2-t^2$, tento polynom má za kořeny $t,-t$ a je ireducibilní v $K(t^2,st)$ protože $t,-t \notin K(t^2,st)$. Tedy $F'/K(t^2,st)$ je tedy jednoduché algebraické rozšíření konečného stupně $[F':K(t^2,st)]=\deg m_{t,K(t^2,st)} = 2$.
\end{enumerate}

\end{document}

