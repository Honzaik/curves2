\documentclass[12pt, a4paper]{article}
\usepackage[margin=1in]{geometry}
\usepackage[utf8x]{inputenc}
\usepackage{indentfirst} %indentace prvního odstavce
\usepackage{mathtools}
\usepackage{amsfonts}
\usepackage{amsmath}
\usepackage{amssymb}
\usepackage{graphicx}
\usepackage{enumitem}
\usepackage{subfig}
\usepackage{float}
\usepackage[czech]{babel}
\usepackage{mathdots}
\usepackage{slashbox}

\begin{document}
\begin{center}
\large NMMB538 - DÚ4

\normalsize Jan Oupický
\end{center}
\vspace{1\baselineskip}

\section{}
Předpokládejme $a^2 \neq 4b, b \neq 0$ viz předchozí úkol. Definujme polynom $w(x,y) = y^2 - x^3 + 2ax^2 - x(a^2-4b)$ neboli máme dokázat, že $F$ je dáno $w(u,v)=0$. Tento polynom je Weirstrassův, tedy víme, že je ireducibilní. Chceme ověřit, že v $F$ platí $w(u,v)=0$.

Z minulého úkolu víme ($t=\frac{y}{x}, s=\frac{b-x^2}{x})$, že $K(t,s)=F' \supset F = K(t^2, st) = K(u,v)$. Dále víme, že platí rovnost $s^2 = t^4 - 2at^2 + (a^2-4b)$ v $F'$. Tedy $u = t^2, v = st \implies s = \frac{v}{t}$ v $F'$. Dosadíme-li $\frac{v^2}{t^2}=\frac{v^2}{u} = u^2-2au+(a^2-4b) \implies v^2 = u^3-2au^2+u(a^2-4b)$. Daná rovnost platí v $F'$, ale obsahuje jen prvky z $F$ tedy platí i v $F$. Tudíž platí $w(u,v)=0$ v $F$.

Ukážeme, že $w(x,y)$ je hladký, tedy genus $F$ je 1.
\begin{gather*}
\frac{\partial w}{\partial x} (x,y) = -3x^2 + 4ax - a^2 + 4b\\
\frac{\partial w}{\partial y} (x,y) = 2y\\
\end{gather*}
Spočteme řešení $-3x^2 + 4ax - a^2 + 4b = 0$. Máme řešení $x_{1,2} = \frac{1}{3}(2a\pm \sqrt{a^2+12b})$. Tedy pokud existuje singularita, tak je v bodě $(x_1, 0)$ nebo $(x_2, 0)$. Ověříme opět, zda pro tyto body platí také $w(x,y)=0$. 

Pro případ $w(x_1,0) = 0$: zajímá tedy kdy $- x_1^3 + 2ax_1^2 - x_1(a^2-4b) = x_1(- x_1^2 + 2ax_1 -(a^2-4b)) = 0$. $x_1 = 0 \iff 2a - \sqrt{a^2+12b} = 0 \iff 4b = a^2$ což nejde z předpokladů. Zbývá tedy $- x_1^2 + 2ax_1 -(a^2-4b) = 0$. Pokud dosadíme za $x_1$, tak dostaneme $\frac{-2}{9}\left(a \sqrt{a^2+12 b}+a^2-12 b\right) = 0$, kde řešení musí splňovat $b=0$ nebo $4b = a^2$.
$w(x,y)$ je tedy smooth, tedy je $F$ eliptické funkční těleso, tedy je rodu 1.

Víme, že $F'/F$ je konečné jednoduché algebraické rozšíření. V předchozím úkolu jsme ukázali $K(t,s)\supset K(t^2,st)$ a $K(t,s)=K(t^2,st)(t)$ a že $m_{t,F}(T)=T^2-t^2$. Tento polynom je ireducibilní nad $F$ a jeho kořeny jsou $t,-t \in F'$. Tedy je $t$ separabilní nad $F$ a $F'=F(t)$, tedy $F'/F$ je separabilní a zároveň $-t \in F'$ tedy je $F'/F$ normální $\implies$ Galoisovo.

\section{}
Definujme $w'(x,y)=y^2-x^3-ax^2-bx \in K[x,y], b \neq 0, 4b \neq a^2$. Dále obdobně $w(x,y) = y^2 - x^3 + 2ax^2 - x(a^2-4b) \in K[x,y], b \neq 0, 4b \neq a^2$. 

Máme definováno, že $F' = K(x,y)$ kde $w'(x,y)=0$ a $F' = K(u,v)$, kde $w(u,v)=0$. Oba polynomy $w',w$ jsou Weirstrassovy a pro funkční tělesa dáná těmito polynomy víme, že platí
\begin{gather*}
P'_\infty \in \mathbb{P}_{F'/K}: v_{P'_\infty}(x)=-2, v_{P'_\infty}(y) = -3\\
P_\infty \in \mathbb{P}_{F/K}: v_{P_\infty}(u)=-2, v_{P_\infty}(v) = -3\\
\end{gather*}
Dále spočteme valuace pro $x,y,u,v$ v místech $P'_{(0,0)}, P_{(0,0)}$. $y,v$ nejsou tečny v $(0,0)$ a $(0,0) \in V_w \cap V_{w'}$, takže jejich valuace je 1. Pro $x,u$ to vychází stejně, jelikož oba polynomy mají $mult_y=2$.
\begin{gather*}
P'_{(0,0)} \in \mathbb{P}_{F'/K}: v_{P'_{(0,0)}}(x)=2, v_{P'_{(0,0)}}(y) = 1\\
P_{(0,0)} \in \mathbb{P}_{F/K}: v_{P_{(0,0)}}(u)=2, v_{P_{(0,0)}}(v) = 1
\end{gather*}

\begin{enumerate}[label=(\alph*)]
    \item Dle definice $\text{div}_{F'/K}(x) = \sum\limits_{P \in \mathbb{P}_{F'/K}} v_P(x)P$. Víme, že jediná místa, kde $v_P(x) \neq 0$ jsou $P'_{(0,0)}$ a $P'_\infty$. Takže $\text{div}_{F'/K}(x) = v'_0(x)P'_{(0,0)}+ v'_\infty(x)P'_\infty = 2P'_{(0,0)} - 2P'_\infty$.

    Obdobně pro zbytek:
    \begin{gather*}
    \text{div}_{F'/K}(y) = v'_0(y)P'_{(0,0)}+ v'_\infty(y)P'_\infty = 1P'_{(0,0)} - 3P'_\infty\\
    \text{div}_{F/K}(u) = v_0(u)P_{(0,0)}+ v_\infty(u)P_\infty = 2P_{(0,0)} - 2P_\infty\\
    \text{div}_{F/K}(v) = v_0(v)P_{(0,0)}+ v_\infty(v)P_\infty = 1P_{(0,0)} - 3P_\infty
    \end{gather*}

    \item Použijeme The Fundamental Equality (F.7) a Proposition F.6. Uvažujme nejprve $P = P_\infty$. Víme, že $[F':F]=2$. Dále dle proposition F.6 pro taková místa $P'$ platí $\deg_{F'/K}(P')[K:K]=f(P'|P)\deg_{F/K}(P)$. Víme ale že pro naše $P=P_\infty, P_{0,0}:$ $\deg_{F/K}(P) = 1$. Tedy $f(P'|P)=\deg_{F'/K}(P')$. Dle F.7 tedy máme 3 možnosti:
    \begin{enumerate}
        \item Existují právě 2 místa $P' \in \mathbb{P}_{F'/K}: P'|P, \deg_{F'/K}(P') = 1$ a platí $e(P'|P)=1=f(P'|P)$.
        \item Existuje jedno místo $P' \in \mathbb{P}_{F'/K}: P'|P, \deg_{F'/K}(P') = 1$ a platí $e(P'|P)=2$ a $f(P'|P)=1$.
        \item Existuje jedno místo $P' \in \mathbb{P}_{F'/K}: P'|P, \deg_{F'/K}(P') = 2$ a platí $e(P'|P)=1$ a $f(P'|P)=2$.
    \end{enumerate}

    

    Uvažujme nyní případ $P=P_\infty$. Víme, že $v_P(u)=-2 \implies u^{-2} \in P$. Spočteme $v'_\infty(u) = v'_\infty(\frac{y^2}{x^2}) = 2v'_\infty(y)-2v'_\infty(x) = 2\cdot(-3)-2\cdot(-2) = -2 \implies u^{-2} \in P'_\infty$. 

    Víme, že $P'_\infty \cap F$ je místo $F/K$ a toto místo obsahuje $u^{-2}$. $P_\infty$ je jediné místo $F/K$ co obsahuje $u^{-2}$. Nezbývá tedy než $P'_\infty \cap F = P_\infty \implies P'_\infty | P_\infty$.

    Obdobně $v'_0(u) = v'_0(\frac{y^2}{x^2}) = 2v'_0(y)-2v'_0(x) = 2\cdot 1-2\cdot2 = -2 \implies u^{-2} \in P'_{(0,0)}$. Stejně jako výše tedy platí $P'_{(0,0)} \cap F = P_\infty \implies P'_{(0,0)} | P_\infty$. 

    Pro $P=P_\infty$ tedy máme 2 různá místa $F'/K$ co ho obsahují $(P'_\infty, P'_{(0,0)})$. Platí možnost a) $\implies e(P'_\infty|P)=e(P'_{(0,0)}|P)=1$ a $f(P'_\infty|P)=f(P'_{(0,0)}|P)=1$

    Uvažujme nyní $P = P_{(0,0)}$. Víme, že místo $P' \in \mathbb{P}_{F'/K}: P'|P$ nemůže už být $P'_\infty$ ani $P'_{(0,0)}$ jinak by $P_{(0,0)} = P' \cap F = P_\infty \implies P_\infty = P_{(0,0)} \implies$ spor.

    Chceme místo $P'$, pro které platí $v_{P'}(u) \ge 2$, protože $P'|P \implies v_{P'}(u)\ge v_P(u)=2$. Platí $v_{P'}(u)=v_P'(\frac{y^2}{x^2}) = 2 (v_{P'}(y)-v_{P'}(x)) \ge 2 \iff v_{P'}(y)-v_{P'}(x) \ge 1$. Rozebereme možné hodnoty $v_{P'}(\cdot)$.

    Pokud $v_{P'}(x) < 0 \implies P' = P'_\infty \implies$ spor. Pokud $v_{P'}(x) > 0, v_{P'}(y) > 0 \implies P' = P'_{(0,0)} \implies $ spor. Zbývá tedy $v_{P'}(x) = 0$ nebo $v_{P'}(y)=0$. Druhá možnost nemůže nastat jelikož by neplatilo $v_{P'}(y)-v_{P'}(x) \ge 1$. Hledáme tedy místo $F'/K$, kde $v_{P'}(x) = 0, v_{P'}(y) \ge 1$.

    Chceme tedy najít body $(x', 0) \in V_{w'} \implies x(x^2+ax+b)= 0$. Pokud $x = 0$, tak máme místo $P'_{(0,0)}$, které nemůžeme použít. Chceme tedy místa příslušná zbylým kořenům. Řešení kvadratické rovnice jsou $x_{1,2}=\frac{1}{2}(-a\pm \sqrt{a^2-4b})$. Z přepokladů to pod odmocninou není 0, tedy máme vždy 2 kořeny za předpokladu že existuje daná odmocnina v $K$.

    Označme místa příslušná těmto bodům $(x_1, 0), (x_2, 0) \in  V_{w'}$ jako $P'_1, P'_2$. Máme tedy 2 různá místa stupně $1$ t.ž. $v_{P'}(y)\geq 1, v_{P'}(x)=0 \implies v_{P'}(u)\geq 2$ (jelikož dále $e(P'|P)=1 \implies v_{P'}(u)=2$). Tedy daná místa obsahují $P_{(0,0)}$ a platí $e(P'|P)=f(P'|P)=1$.

    Pokud neexistuje $\sqrt{a^2-4b}$. Tak neexistuje jiné místo stupně $1$ obsahující $y$. Tedy zbývá možnost $P'$ je jediné místo obsahující $P_{(0,0)}$, $P'$ je stupně $2$. Dále se nebudeme tímto případem zabývat.

    \item Pro přehlednost označme $P'_0 \coloneqq P'_{(0,0)}, P_0 \coloneqq P_{(0,0)}$. Dle b) tedy $P'_0, P'_\infty | P_\infty$ a $P'_1, P'_2 | P_0$. Víme, že $\text{div}_{F/K}(u) = 2P_0 - 2P_\infty$. Dále jsme zjistili, že jediná místa $P' \in \mathbb{P}_{F'/K}$, kde $v_{P'}(u) < 0$ jsou $P'_0$ a $P'_\infty$. Kdyby totiž existovalo jiné místo $F'/K$ t.ž. $v_{P'}(u)<0 \implies P' \cap F = P \in \mathbb{P}_{F/K}$ neboli místo t.ž. $v_P(u) < 0$ což musí být $P_\infty$ a jiná místa co ho dělí už nejsou, tedy spor. Stejně tak jiná místa $P' \in \mathbb{P}_{F'/K}: v_{P'}(u) > 0$ než $P'_1, P'_2$ nejsou.

    Z toho plyne $\text{div}_{F'/K}(u) = 2P'_1 + 2P'_2 - 2P'_0 - 2P'_\infty$.

    Spočteme $\text{Con}_{F'/F}(\text{div}_{F/K}(u)) = 2(\sum\limits_{P'|P_0}1\cdot P') -2(\sum\limits_{P'|P_\infty}1\cdot P') = 2(P'_1+P'_2)-2(P'_0+P'_\infty)$.

    Rovnost tedy platí.

    \item Máme $\text{Con}_{F'/F}(P_\infty) = \sum\limits_{P'|P}1\cdot P' = P'_0 + P'_\infty$. Potom $\deg_{F'/K}(P'_0+P'_\infty) = \deg_{F'/K}(P'_0) + \deg_{F'/K}(P'_\infty) = 1+1=2$. První $\deg$ značí stupeň divisoru a druhý deg je stupeň místa.

    Na druhé straně $\deg_{F/K}{P_\infty} = 1$ a $[F':F] = 2$ tedy rovnost platí.

    \item $\text{div}_{F'/K}(x) = 2P'_0 - 2P'_\infty$. $P'_0 \cap F = P_\infty = P'_\infty \cap F$ a $f(P'_\infty|P_\infty)=1=f(P'_0|P_\infty)$ jak jsme zjistili výše. Hodnota $\text{N}_{F'/F}(\text{div}_{F'/K}(x)) = \text{N}_{F'/F}(2P'_0 - 2P'_\infty) = 2(1P_\infty)-2(1P_\infty) = 0$.

    $[F':F]=2$ a $F'/F$ je Galoisovo rozšíření, tedy $|\text{Gal}(F'|F)|=2$. Zřejmě jeden automorfismus je identita, tedy $\sigma_1(x)=x$. 

    V minulém úkolu, jsme dokázali, že $F'=K(s,t) = F(t)$ a že $min_{t,F}(T)=T^2-t^2$. Tedy prvky $\text{Gal}(F'|F)$ permutují kořeny zmíněného minimálního polynomu ($t, -t$). Víme, že $\sigma_1 = id \implies \sigma_2(t) = -t$.

    Nyní vyjádříme $x \in F'$ v bázi $(1,t)$ nad $F$. Použijeme vzorec z minulého úkolu pro výpočet $x$ pomocí $t$ a $s$. Výsledkem je:
    \begin{gather*}
    x = \frac{2bt^6-2abt^4}{(t^4-at^2)^2-(st)^2t^2} + t \cdot \frac{-2bst^3}{(t^4-at^2)^2-(st)^2t^2}
    \end{gather*}

    Díky tomu můžeme spočítat tedy $\sigma_2(x)$:
    \begin{gather*}
    \sigma_2(x) = \frac{2bt^6-2abt^4}{(t^4-at^2)^2-(st)^2t^2} + \sigma_2(t) \cdot \frac{-2bst^3}{(t^4-at^2)^2-(st)^2t^2} = \\
    \frac{2bt^6-2abt^4}{(t^4-at^2)^2-(st)^2t^2} + (-t) \cdot \frac{-2bst^3}{(t^4-at^2)^2-(st)^2t^2}
    \end{gather*}

    Po dosazení za $t,s$ vyjde, že $\sigma_2(x)=bx^{-1}$

    Dle S.12 tedy $N_{F'/F}(x)=\sigma_1(x)\cdot \sigma_2(x) = x \cdot bx^{-1} = b$ a zřejmě $\text{div}_{F/K}(b) = 0$, jelikož $b \in K$. Rovnost tedy platí.

    \item Máme $F' \supset F \supset K(v), [F':F]=2$. Dále platí $[F:K(v)]=3$, jelikož polynom $w(x,y)$ z 1) dává minimální polynom $u$ nad $K(v)$. $F/K(v)$ je tedy algebraické konečného stupně.

    Dle lemma F.9 je tedy $\text{N}_{F'/K(v)}(\text{div}_{F'/K}(x)) = 0$, protože dle e) $\text{N}_{F'/F}(\text{div}_{F'/K}(x)) = 0$ a zřejmě $\text{N}_{F/K(v)}(0) = 0$. 

    Obdobně použitím proposition S.13 platí $N_{F'/K(v)}(x) = b$, jelikož dle e) $N_{F'/F}(x) = b$ a $N_{F/K(v)}(b) = b$. Poté $\text{div}_{K(v)/K}(b) = 0$
\end{enumerate}

\section{}
Proposition F.13 říká v našem případě:
\begin{gather*}
\deg_{F'/K}(\text{Con}_{F'/F}(P)) = \frac{[F':F]}{[K':K]}\cdot \deg_{F/K}(P) = \frac{2}{1} \cdot 1 = 2 \implies\\
\text{Con}_{F'/F}(P) = \sum\limits_{P'|P}e(P'|P)P' \implies \deg_{F'/K}(\text{Con}_{F'/F}(P)) = \sum\limits_{P'|P}e(P'|P)\deg_{F'/K}(P') \implies\\
\sum\limits_{P'|P}e(P'|P)\deg_{F'/K}(P') = 2
\end{gather*}
Máme tedy 3 možnosti:

\begin{enumerate}
    \item Máme $P_1 \neq P_2: P_1, P_2 | P$. Poté už musí platit $\deg_{F'/K}(P_1)=1=\deg_{F'/K}(P_2)$ a $e(P_1|P)=1=e(P_2|P)$.
    \item Nebo jediné $P' | P$, pro které buď $e(P'|P)=2$ a následně musí být $\deg_{F'/K}(P')=1$,
    \item nebo $e(P'|P)=1$ a následně musí $\deg_{F'/K}(P')=2$.
\end{enumerate}

Dle proposition F.6 platí (v našem případě $K'=K$), že pokud $P'|P$, pak $\deg_{F'/K}(P') = f(P'|P)\cdot \deg_{F/K}(P)$. Jelikož $f(P'|P) \geq 1$ a předpokládáme, že $\deg_{F'/K}(P')=1$, nezbývá nic jiného, než $\deg_{F/K}(P) = 1$.

V předpokladech proposition F.6 je pouze, že $F'/F$ je algebraické rozšíření. Podíváme-li se ale na důkaz podtvrzení $\deg_{F'/K}(P')[K':K] = f(P'|P)\deg_{F/K}(P)$, tak se v důkazu nikde algebraičnost nepoužívá. Jediný bod, kde se využívá znalost tvrzení dokázaných pro algebraická, je $O_P \cap P' = P$. Toto ale platí obecně pro $P'|P$:
\[O_P \cap P' = (P \cup O^*_P) \cap P' = (P \cap P') \cup (O^*_P \cap P') \]

Zřejmě $P \cap P' = P$. Chceme tedy zjistit jestli $O^*_P \cap P' = \O$. Uvažujme pro spor $x \in O^*_P \cap P' \implies x \in F, v_{P'}(x) > 0$. Zvolme uniformující element $P$ označme ho $t$. Poté $x = u t^k, u \in O^*_P, k \in Z$. Ale $0<v_{P'}(x) = v_{P'}(ut^k)=v_{P'}(u)v_{P'}(t^k)$, ale zřejmě $k = 0$ tedy $0<v_{P'}(x) = v_{P'}(u)v_{P'}(1)= 0$, což je spor. Tedy je průnik prázdný.

\end{document}

