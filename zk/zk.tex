\documentclass[12pt, a4paper]{article}
\usepackage[margin=1in]{geometry}
\usepackage[utf8x]{inputenc}
\usepackage{indentfirst} %indentace prvního odstavce
\usepackage{mathtools}
\usepackage{amsfonts}
\usepackage{amsmath}
\usepackage{amssymb}
\usepackage{graphicx}
\usepackage{enumitem}
\usepackage{subfig}
\usepackage{float}
\usepackage[czech]{babel}
\usepackage{mathdots}
\usepackage{slashbox}

\newcommand{\qed}{\hfill\square}

\begin{document}
\begin{center}
\large NMMB538 - Zkouška

\normalsize Jan Oupický
\end{center}
\vspace{1\baselineskip}

\textbf{Lemma Q.1.} \textit{Proof:}

Denote $h = x_2^2 - f(x_1)$ and assume $h = u \cdot v$ where $u, v \in \bar{K}[x_1,x_2]$. 

First assume $u,v \in \bar{K}[x_1,x_2] \setminus \bar{K}[x_1]$ i.e. $\deg_{x_2}(u) > 0, \deg_{x_2}(v) > 0$. Because $\deg_{x_2}(u) + \deg_{x_2}(v) = \deg_{x_2}(h) = 2 \implies \deg_{x_2}(u) = 1 = \deg_{x_2}(v)$. W.l.o.g assume $lc_{x_2}(u)=1=lc_{x_2}(v)$, we can do that since $lc_{x_2}(h)=1$. Therefore we can write $u = x_2 - s_1$ and $v = x_2 - s_2$ where $s_1, s_2 \in \bar{K}[x_1]$. This gives us
\[
x_2^2 - f(x_1) =h = (x_2 - s_1)(x_2 - s_2) = x_2^2 - (s_1+s_2)x_2 + s_1 s_2
\]
So it must hold that $s_1 = -s_2$ and then $h = x_2^2 + s_1(-s_1) \implies f(x_1) = s_1^2$. 

Now assume w.l.o.g $u \in \bar{K}[x_1]$.We compare the leading coefficients.
\[
1 = lc_{x_2}(h) = lc_{x_2}(u) \cdot lc_{x_2}(v) = u \cdot lc_{x_2}(v)
\]
This shows that $u$ must be invertible in $\bar{K}[x_1,x_2] \implies u \in \bar{K}^*$. In other words $h$ is absolutely irreducible. 

$\qed$

\textbf{Sublemma Q.3.5} \textit{Let $F/K$ be an algebraic function field, $char(K) \neq 2$, that is given by $y^2 = f(x)$, $f$ being a quaternary polynomial that posseses a simple root. Let $P \in \mathbb{P}_{F/K}$. If $x \notin P$ or $y \notin P$, then $x,y \notin P$ and $2v_P(x)=v_P(y)$}.

\textit{Proof:}
In $F$ it holds $y^2 = f(x)$ by definition which implies that for every $P \in \mathbb{P}_{F/K}$ $v_P(y^2) = 2v_P(y)=v_P(f(x))$.

Assume $v_P(x) < 0 \leq v_P(y)$. By properties of valution we have $\deg(f)v_P(x) = v_P(f(x)) = 2v_P(y) \implies 2v_P(x)=v_P(y)$ and by assumption $v_P(y) > v_P(x) \implies 2v_P(x) > v_P(x) \iff v_P(x) > 0$. That's a contradiction.

Now assume $v_P(x) \geq 0 > v_P(y)$. $v_P(x) \geq 0 \implies v_P(f(x)) \geq 0$ then $0 \leq v_P(f(x)) = 2v_P(y) < 0$ which is again a contradiction.

We have proven $v_P(x) < 0 \iff v_P(y) < 0$. Therefore we have the equality $4v_P(x) = 2v_P(y) \iff 2v_P(x) = v_P(y)$ assuming $v_P(x) < 0$ or $v_P(y) < 0$. 

$\qed$

\textbf{Lemma Q.4.} \textit{Proof:}
By sublemma Q.3.5 we know, that if $P \in \mathbb{P}_{F/K}: x^{-1} \in P \implies y^{-1} \in P$ and $2v_P(x)=v_P(y)$. This proves $(y)_{-}=2(x)_{-}$ ($x^{-1},y^{-1}$ "share" places and the valuation is 2:1).

Let's first assume that $f$ possesses a multiple root. Therefore $f(x_1) = (x_1-\alpha)^2g(x_1)$ where $\deg(g) = 2$ and $g$ is not a square. By Q.3 $F$ is given by $z^2 = g(x)$ i.e. $F=K(x,z)$. $[F:K(x)]=2$ since $min_{z,K(x)}(T) = T^2 - g(x)$, that polynomial has $z$ as a root in $F$ and it is absolutely irreducible (as a polynomial in $K[x,T]$) since $g$ is not a square. We can then assume $\bar{K}=K$ since $[F:\bar{K}(x)] = 2$ (same polynomial) and $[F:K(x)] = [F:\bar{K}(x)][\bar{K}:K] = 2 \implies [\bar{K}:K] = 1$.

Then we know $\deg((x)_{-}) = [F:K(x^{-1})] = [F:K(x)] = 2$ i.e. $\deg(D)=2$.

Now assume $f$ is separable. We can then use the same argument for $K=\bar{K}$ since $min_{y,K(x)}(T) = T^2-f(x)$ and by Q.1 this one is also absolutely irreducible. $F=K(x,y) \implies [F:K(x)] = [F:\bar{K}(x)][\bar{K}:K] = 2 \implies [\bar{K}:K]=1$. And again $\deg(D)=2$.

Now let's prove that the genus is at most 1. Since $\deg(D)=2 \implies \sum\limits_{P: x^{-1}\in P} v_P(x)\deg(P) = 2$ means we have 2 possibilities (we are assuming $K=\bar{K}$ which implies $\forall P \in \mathbb{P}_{F/K}: \deg(P)=1$:
\begin{enumerate}
    \item There exists a unique place $P_\infty: v_{P_\infty}(x)=-2, v_{P_\infty}(y)=-4$ and $D = 2P_\infty$
    \item There are 2 distinct places $P,Q$ s.t. $v_P(x)=-1=v_Q(x), v_P(y)=-2=v_Q(y)$ and $D = P + Q$.
\end{enumerate}

In both cases we can see that for $k \geq 2: \{1,x,\dots,x^k, y, yx, \dots, yx^{k-2}\} \subset \mathcal{L}(kD)$ because $(x^k) + kD = k((x)_{+}-(x)_{-})+k(x)_{-} = k(x)_{+} \geq 0$ and also $(y)_{-}=2(x)_{-}$ so it holds if we substitute $x^2$ for $y$. This subset is linearly indepent over $K$ because $y$ cannot be expressed as a linear combination of $x^i$ since $f$ has one simple root (if $f(x)=g^2(x) \implies y=g(x)$). The set also contains $2k$ elements. Therefore $l(kD) \geq 2k$. 

We know that for a sufficiently large $k$ (if $l(kD) \geq 2g-1$, $g$ genus) we have $l(kD) = \deg(kD)-g+1$ having $\deg(kD)=2k, l(kD) \geq 2k \implies 0 \leq l(kD)-\deg(kD) = -g + 1 \iff g \leq 1$.

$\qed$

\textbf{Proposition Q.5.} \textit{Proof:}
Denote $ax^3+bx^2+cx+d = g(x)=f(x)-x^4$. First we will prove that for both $z \in Z = \{y+x^2, y-x^2\}: [F:K(z)]=2$. Denote $z_1=y+x^2, z_2=y-x^2$ We have tower of field extensions $F \supseteq K(z_1,z_2) \supseteq K(z_1)$:
\begin{gather*}
[F:K(z_1)] = [F:K(z_1,z_2)][K(z_1,z_2):K(z_1)]
\end{gather*}
Clearly $K(x^2,y) \supseteq K(z_1,z_2)$ and conversely $x^2 = \frac{z_1-z_2}{2}, y = \frac{z_1+z_2}{2} \implies K(z_1,z_2) \supseteq K(x^2,y)$ therefore $K(z_1,z_2)=K(x^2,y)$.

Also $K(x^2,y) =F$ since $x = \frac{y^2-x^4-bx^2-d}{ax^2+c} \in K(x^2,y)$ if $a\neq 0$ and $c\neq 0$ ($F=K(x,y)$). If $a=0=c$ then $F$ is given by $y^2=g'(x^2)$ where $g'(x)=x^2+a'x+b$ separable. Then $K(x^2,y)=F$ as well.

Choose $P \leq D$ a place. We will prove that for at least one $z \in Z: v_P(z) < v_P(x)$.
\begin{gather*}
y^2 = f(x) \iff y^2 - x^4 = g(x), 0 \leq \deg(g) \leq 3 \implies\\
(y+x^2)(y-x^2)=g(x) \implies v_P(y+x^2)+v_P(y-x^2) = v_P(g(x)) = \deg(g)v_P(x)
\end{gather*}
Denote $z_1 = y+x^2, z_2=y-x^2$. Assume to contrary $v_P(z_1) \geq v_P(x)$ and $v_P(z_2) \geq v_P(x)$. We will look at all possible cases.

$\deg(g)=3$: $3v_P(x)=v_P(z_1)+v_P(z_2) \geq v_P(x)+v_P(x) \implies v_P(x) > 0$ which is a contradiction since $v_P(x)<0$.

$\deg(g)=2$: $2v_P(x)=v_P(z_1)+v_P(z_2)$. 

First consider $v_P(x)=v_P(z_1)=v_P(z_2)$ and $v_P(x)=-2$, then $2P = (x)_{-}= (z_1)_{-} = (z_2)_{-}$. Then $z_1, z_2 \in \mathcal{L}(D)$ but then also $\frac{z_1+z_2}{2} = y \in \mathcal{L}(D)$ which is a contradiction since $(y)_{-}=2(x)_{-}$.

Now let's assume $v_P(x)=-1$ and then there also exist a different place $Q$ s.t. $v_Q(x)=v_Q(z_1)=v_Q(z_2)=-1$. Since $\deg((z)_{-})=2$ then again $P+Q = (x)_{-}= (z_1)_{-} = (z_2)_{-}$ and we have the same contradiction.

We have proven that $v_P(x)=v_P(z_1)=v_P(z_2)$ is impossible therefore for one $z$ it must hold $v_P(z)<v_P(x)$.

$\deg(g)=1$: $v_P(x)=v_P(z_1)+v_P(z_2)$. First assume $v_P(x)=-2$. And also $v_P(z_1)=v_P(z_2)=-1$. This is impossible since $\deg((z)_{-})=2$ but for every other place $Q \neq P: 0 = v_Q(z_1)+v_Q(z_2) \implies v_Q(z_1)=-v_Q(z_2)$. If there was $Q_1: v_{Q_1}(z_1)=-1 \implies v_{Q_1}(z_2)=1$ and $Q_2: v_{Q_2}(z_2)=-1 \implies v_{Q_2}(z_1)=1$. Then for some $P_1,P_2$ places of degree 1: $(z_1)=(P_1+Q_2)-(P+Q_1), (z_2)=(P_2+Q_1)-(P+Q_2)$. Set $D'=P+Q_1+Q_2$ then $z_1,z_2 \in \mathcal{L}(D')$ and as before this means that $y \in \mathcal{L}(D')$ which is a contradiction.

Now if $v_P(x)=-1$ then either $v_P(z) < v_P(x)$ for a $z \in Z$ or w.l.o.g $v_P(z_1)=-1$ and $v_P(z_2)=0$. Then also assume first $v_Q(z_1)=-1 \implies v_Q(z_2)=0$. But since $\deg((z)_{+})=2$ there must be a place $P'$ s.t. $v_{P'}(z_2)>0$ and $v_{P'}(x)=0$ since $P'\neq P,Q$ but it must be $v_{P'}(z_1) < 0$. This again contradicts the degree of the divisor. 

If $v_Q(z_1)=0$ and $v_Q(z_2)=-1$. Then again there must be a place $P_1$ s.t. $v_{P_1}(z_1)=-1$ and $v_{P_1}(z_2)=1$ and a place $v_{P_2}(z_2)=1 \implies v_{P_2}(z_1)=-1$. This also contradicts divisor degree.

The last case is $\deg(g)=0$: $v_P(z_1)=-v_P(z_2) \implies (z_1) = -(z_2)$. First assume $v_P(z_1)=0 \implies v_P(z_2)=0$ and same for $v_Q$ ($Q$ not necessarly different from $P$). Then there exist places $P_1,P_2,Q_1,Q_2 \neq P,Q$ s.t. $(z_1)=P_1+P_2-(Q_1+Q_2), (z_2)=-(z_1)$. Set $D' = P_1+P_2+Q_1+Q_2$ then $z_1,z_2 \in \mathcal{L}(D')$ but also $y \in \mathcal{L}(D')$ which is again contradiction since $(y)_{-}=2(x)_{-}$.

If $v_P(z_1)=1 \implies v_P(z_2)=-1$. There exists another place $P'$ s.t. $v_{P'}(z_1)=1 \implies v_{P'}(z_2)=-1$. There must be again two places $Q_1,Q_2$ s.t. $(z_1)=P+P'-(Q_1+Q_2), (z_2) = -(z_1)$. Put $D' = P+P'+Q_1+Q_2$ then again $y \in \mathcal{L}(D')$ which is a contradiction.
\\

We have proven that for each place $P \leq D$ at least one $z \in Z$ must have $v_P(z)<v_P(x)$. This shows also that $(x)_{-}=P+Q$ for distinct $P,Q$. If $P=Q$ then $v_P(x)=-2 \implies v_P(z)\leq -3$ which contradicts $[F:K(z)]=2$. Since $\deg((z)_{-})=2$ and $v_P(z) < -1$ it must be that $(z)_{-}=2P, (z')_{-}=2Q$ for $z,z' \in Z$. Since we have not distinguished $P$ and $Q$ we can say $(z_1)_{-}=2P$ and $(z_2)_{-}=2Q$.



$\qed$


\textbf{Theorem Q.6.} \textit{Proof:}
Assume genus 0. There exists $t \in F$ s.t. $(t)=P-Q$ and also $(t^{-1})=-(t)=Q-P$. Also $l(D)=\deg(D)+1 = 3$. 

$t \in \mathcal{L}(D)$ since $(t)+D = P-Q+P+Q=2P \geq 0$. Also $t^{-1} \in \mathcal{L}(D): (t^{-1})+D=Q-P+P+Q = 2Q \geq 0$. $t$ and $t^{-1}$ are linearly independent since $t \notin K$. This means $\{1,t,t^{-1}\}$ is a basis of $\mathcal{L}(D)$.

$x \in \mathcal{L}(D) \implies x = c_0+c_1t+c_2t^{-1}$ for some $c_i \in K$. This is equivalent to saying $tx = u(t), u(t) \in K[t], \deg(u)=2$.

In the same way we see $\{1,t,t^{-1},t^2,t^{-2}\}$ for a basis of $\mathcal{L}(2D)$. Again $y \in \mathcal{L}(2D): (y) + 2D = (y)_{+} - (y)_{-} + 2(x)_{-} = (y)_{+} \geq 0$. This means $t^2y = v(t)$ where $v(t) \in K[t], \deg(v)=4$.

$y^2 = f(x) \iff t^4y^2=t^4f(x)$. Substitute $yt=v(t)$ and $xt^2=u(t)$ then we have equality $v^2(t)=t^4f(\frac{u(t)}{t})$. $f$ is a polynomial of degree $4$ therefore it has up to $4$ different roots $1 \leq i \leq 4: \alpha_i \implies v^2(t)=t^4(\frac{u(t)}{t}-\alpha_1)(\frac{u(t)}{t}-\alpha_2)(\frac{u(t)}{t}-\alpha_3)(\frac{u(t)}{t}-\alpha_4)$ we can rewrite this as 
\begin{gather*}
v^2(t)=(u(t)-t\alpha_1)(u(t)-t\alpha_2)(u(t)-t\alpha_3)(u(t)-t\alpha_4)
\end{gather*}
$v^2(t)=v(t)v(t)$ is a polynomial of degree $8$, which has at most 4 different roots. Also $u(t)-t\alpha_i$ is a polynomial of degree 2. There exist at most two $\alpha \in K$ s.t. $u(t)-t\alpha$ has a root of multiplicity 2. This is because the root of a quadratic polynomial has multiplicity 2 when the discriminant D is 0. If $g(x)-\alpha x=ax^2+(b-\alpha)x+c \implies 0 = \text{D} = (b-\alpha)^2-4ac \iff \alpha = \pm 2\sqrt{ac}+b$.

Also if $i\neq j: \alpha_i \neq \alpha_j$ the polynomials $u(t)-t\alpha_i$ and $u(t)-t\alpha_j$ do not have common roots. This means that if $1 \leq i \leq 4: \alpha_i$ are all different then $v^2(t)$ has at least $6 = 2+2+1+1$ different roots. This a contradiction.

Therefore if genus is $0$ then $f$ cannot be separable. We have shown that genus of $F$ is 0 or 1, this means that for $f$ separable we must have genus 1.

Denote $w=x_2^2-f(x_1)$.
\begin{gather}
\frac{\partial w}{\partial x_1} = -f'(x_1), \frac{\partial w}{\partial x_1} = 2x_2
\end{gather}
For a singularity $\alpha = (\alpha_1, \alpha_2), \alpha_2$ must be $0$ and $\alpha_1$ must be a root of $f(x_1)$ and also of $f'(x_1)$ this is true iff $f(x_1)$ is separable. If $f(x_1)$ is not separable then it shares a common root $\alpha_1$ with $f'(x_1)$ and this gives us singularity at $(\alpha_1, 0)$. This proves the rest of the theorem.

$\qed$

\textbf{Theorem Q.7.} \textit{Proof:}
Denote $D = P + Q$ a divisor. Due to genus being 1 $\forall k\geq 1: l(kD) = 2k$. $l(D)=2$ and that means there exists $x \notin K$ s.t. $\{1,x\}$ is a basis of $\mathcal{L}(D)$ and also $(x)_{-} \leq P+Q$. Then $(x)^2 = 2(x) = 2(x)_{+}-2(x)_{-} \implies x^2 \in \mathcal{L}(2D),\{1,x,x^2\}$ is linearly independent in $\mathcal{L}(2D)$ but $l(2D)=4$ that means there exists $y \in \mathcal{L}(2D)\setminus \mathcal{L}(D)$ such that $\{1,x,x^2,y\}$ is a basis of $\mathcal{L}(2D)$. 

Denote $B = \{1,x,x^2,x^3,x^4,y,yx,yx^2,y^2\}$, clearly $B \subseteq \mathcal{L}(4D), l(4D)=8$ and $|B|=8 \implies 1 \leq i \leq 8: \exists a_i \in K:$
\begin{gather*}
y^2 = a_1y + a_2yx + a_3yx^2 + a_4x^4 + a_5x^3 + a_6x^2 + a_7x+ a_8
\end{gather*}

Denote $C = \{1,x,x^2,x^3,y,yx\}$. $C$ is a basis of $\mathcal{L}(3D)$, $C \cup \{yx^2,y^2\}$ is also a basis of $\mathcal{L}(4D)$. If $a_4=0$ that would be a contradiction to $C \cup \{yx^2,y^2\}$ being a basis of $\mathcal{L}(4D)$ since $y^2$ would be a linear combination of 7 elements.

Now we make a substitution $y \rightarrow y - \frac{a_1+a_2x+a_3x^2}{2}$. This gives us form:
\begin{gather*}
y^2 =b_1x^4 + b_2x^3 + b_3x^2 + b_4x+ b_5
\end{gather*}
where $b_1 = a_4+\frac{a_3^2}{4}$. If $b_1=0$ then $y^2$ would be a linear combination of elements in $\mathcal{L}(3D)$ but 

$\qed$

\textbf{Theorem Q.8.} \textit{Proof:}
Denote $f(x) = g(x^2)$. $F$ is EFF therefore genus is 1 and there exists a place of degree $1$. Also $K$ can be assumed algebraically closed.

If $g(x)$ has a multiple root $\alpha$, then $f(x)=g(x^2)$ has also a multiple root because $g(x)=(x-\alpha)^2 \implies g(x^2) = (x-\sqrt{\alpha})^2(x+\sqrt{\alpha})^2$. Set $z = \frac{y}{x-\sqrt{\alpha}}$. Then $F$ is given by $z^2=(x+\sqrt{\alpha})^2$. This means $x \in K(z)$ and from the definition of $y$ also $y \in K(z)$ which means $F$ has genus 0, a contradiction.

From now on we can assume $g(x)$ has 2 distinct roots. If $f(x)=g(x^2)$ would have a multiple root then it's genus would not be $1$ by Q.6. So we can assume $g(x^2)$ separable. 

First we will prove the second part of the theorem. We have shown that $g(x^2)$ must be separable. Therefore by Q.5 we have places of degree 1 ($P\neq Q$), $(x)_{-} = P+Q$ and $(y+x^2)_{-}=2Q, (y-x^2)_{-} = 2P$.
\begin{gather*}
y^2=g(x^2) = x^4+2bx^2+c \iff y^2-(x^4-2bx^2-b^2) = c - b^2 \implies \\
(y-(x^2+b))(y+(x^2+b))=c-b^2
\end{gather*}

Since $g(x^2)$ is separable $g(x)$ must have simple roots. If $g(x)$ has a multiple root then it's discriminant is 0 and that happens iff $c-b^2=0$. So we know $0 \neq c-b^2 \in K$.

\begin{gather*}
0=v_P(c-b^2)=v_P(y-(x^2+b)) + v_P(y+(x^2+b))\\
v_P(y-(x^2+b)) = v_P(y-x^2) = -2 \implies v_P(y+(x^2+b)) = 2
\end{gather*}

Similarly we can show $v_Q(y-(x^2+b)) = 2$ and $v_Q(y+x^2+b)=-2$. 

Since $\deg((y+x^2+b)_{+}) = \deg((y+x^2+b)_{-}) = \deg((y+x^2)_{-}) = 2 \implies \text{div}(y+x^2+b) = 2P - 2Q$ and similarly $\text{div}(y+x^2+b) = 2Q - 2P$.

We have proven the last part of the theorem. Now let's prove the equivalence.

As we have shown before. We can assume $g(x^2)$ separable and then we have involution $P-Q$ as shown above since $2P-2Q = (t)$ for $t \in F$. We only have to show that $P-Q \neq (t)$ for some $t \in F$. 

If $t \in F$ s.t. $(t) = P - Q \implies \deg((t)_{+}) = 1 = [F:K(t)]$ and that would be contradiction with F being EFF.

Now we assume we have involution. We can always find $t \in F \setminus K$ s.t. $(t) = 2P-2Q$ where $P,Q$ distinct places of degree 1 and $P-Q$ is involution.

Then $l(2P) = 2 = l(2Q) \implies \{1,t\}$ is a basis of $\mathcal{L}(2P)$ and $\{1,t^{-1}\}$ is a basis of $\mathcal{L}(2Q)$.




\end{document}