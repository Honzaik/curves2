\documentclass[12pt, a4paper]{article}
\usepackage[margin=1in]{geometry}
\usepackage[utf8x]{inputenc}
\usepackage{indentfirst} %indentace prvního odstavce
\usepackage{mathtools}
\usepackage{amsfonts}
\usepackage{amsmath}
\usepackage{amssymb}
\usepackage{graphicx}
\usepackage{enumitem}
\usepackage{subfig}
\usepackage{float}
\usepackage[czech]{babel}
\usepackage{mathdots}
\usepackage{slashbox}

\newcommand{\qed}{\hfill\square}

\begin{document}
\begin{center}
\large NMMB538 - Zkouška

\normalsize Jan Oupický
\end{center}
\vspace{1\baselineskip}

\textbf{Lemma Q.1.} \textit{Proof:}

Denote $h = x_2^2 - f(x_1)$ and assume $h = u \cdot v$ where $u, v \in \bar{K}[x_1,x_2]$. 

First assume $u,v \in \bar{K}[x_1,x_2] \setminus \bar{K}[x_1]$ i.e. $\deg_{x_2}(u) > 0, \deg_{x_2}(v) > 0$. Because $\deg_{x_2}(u) + \deg_{x_2}(v) = \deg_{x_2}(h) = 2 \implies \deg_{x_2}(u) = 1 = \deg_{x_2}(v)$. W.l.o.g assume $lc_{x_2}(u)=1=lc_{x_2}(v)$, we can do that since $lc_{x_2}(h)=1$. Therefore we can write $u = x_2 - s_1$ and $v = x_2 - s_2$ where $s_1, s_2 \in \bar{K}[x_1]$. This gives us
\[
x_2^2 - f(x_1) =h = (x_2 - s_1)(x_2 - s_2) = x_2^2 - (s_1+s_2)x_2 + s_1 s_2
\]
So it must hold that $s_1 = -s_2$ and then $h = x_2^2 + s_1(-s_1) \implies f(x_1) = s_1^2$. 

Now assume w.l.o.g $u \in \bar{K}[x_1]$. We compare the leading coefficients.
\[
1 = lc_{x_2}(h) = lc_{x_2}(u) \cdot lc_{x_2}(v) = u \cdot lc_{x_2}(v)
\]
This shows that $u$ must be invertible in $\bar{K}[x_1,x_2] \implies u \in \bar{K}^*$. In other words $h$ is absolutely irreducible. 

$\qed$

\textbf{Sublemma Q.3.5} \textit{Let $F/K$ be an algebraic function field, $char(K) \neq 2$, that is given by $y^2 = f(x)$, $f$ being a quaternary polynomial that posseses a simple root. Let $P \in \mathbb{P}_{F/K}$. If $x \notin P$ or $y \notin P$, then $x,y \notin P$ and $2v_P(x)=v_P(y)$}.

\textit{Proof:}
In $F$ it holds $y^2 = f(x)$ by definition which implies that for every $P \in \mathbb{P}_{F/K}$ $v_P(y^2) = 2v_P(y)=v_P(f(x))$.

Assume $v_P(x) < 0 \leq v_P(y)$. By properties of valution we have $\deg(f)v_P(x) = v_P(f(x)) = 2v_P(y) \implies 2v_P(x)=v_P(y)$ and by assumption $v_P(y) > v_P(x) \implies 2v_P(x) > v_P(x) \iff v_P(x) > 0$. That's a contradiction.

Now assume $v_P(x) \geq 0 > v_P(y)$. $v_P(x) \geq 0 \implies v_P(f(x)) \geq 0$ then $0 \leq v_P(f(x)) = 2v_P(y) < 0$ which is again a contradiction.

We have proven $v_P(x) < 0 \iff v_P(y) < 0$. Therefore we have the equality $4v_P(x) = 2v_P(y) \iff 2v_P(x) = v_P(y)$ assuming $v_P(x) < 0$ or $v_P(y) < 0$. 

$\qed$

\textbf{Lemma Q.4.} \textit{Proof:}
By sublemma Q.3.5 we know, that if $P \in \mathbb{P}_{F/K}: x^{-1} \in P \implies y^{-1} \in P$ and $2v_P(x)=v_P(y)$. This proves $(y)_{-}=2(x)_{-}$ ($x^{-1},y^{-1}$ "share" places and the valuation is 2:1).

Let's first assume that $f$ possesses a multiple root. Therefore $f(x_1) = (x_1-\alpha)^2g(x_1)$ where $\deg(g) = 2$ and $g$ is not a square (since $f$ has a simple root). By Q.3 $F$ is given by $z^2 = g(x)$ i.e. $F=K(x,z)$. $[F:K(x)]=2$ since $min_{z,K(x)}(T) = T^2 - g(x)$, that polynomial has $z$ as a root in $F$ and it is absolutely irreducible (as a polynomial in $K[x,T]$) since $g$ is not a square. This also means it is irreducible over $\tilde{K}$. We can then assume $\tilde{K}=K$ since $F \supseteq \tilde{K}$ and $2 = [F:K(x)] = [F:\tilde{K}(x)] [\tilde{K}(x):K(x)]$ ($[\tilde{K}(x):K(x)]=[\tilde{K}:K]$). Also $[F:\tilde{K}(x)]=2$ (same polynomial) which implies $[\tilde{K}:K] = 1$.

Then we know $\deg((x)_{-}) = [F:K(x^{-1})] = [F:K(x)] = 2$ i.e. $\deg(D)=2$.

Now assume $f$ is separable. We can then use the same argument for $K=\tilde{K}$ since $min_{y,K(x)}(T) = T^2-f(x)$ and by Q.1 this one is also absolutely irreducible. $F=K(x,y) \implies 2=[F:K(x)] = [F:\tilde{K}(x)][\tilde{K}:K] = 2 \implies [\tilde{K}:K]=1$. And again $\deg(D)=2$.

We can see that for $k \geq 2: \{1,x,\dots,x^k, y, yx, \dots, yx^{k-2}\} \subset \mathcal{L}(kD)$ because $(x^k) + kD = k((x)_{+}-(x)_{-})+k(x)_{-} = k(x)_{+} \geq 0$ and also $(y)_{-}=2(x)_{-}$ so it holds if we substitute $x^2$ for $y$. To show that this set is linearly independent is equivalent to say that $p(x)+yq(x) = 0 \in F \iff p(x)+yq(x) \in (y^2-f(x)) \iff p(x)=0,q(x)=0$ for $p(x),q(x) \in K(x)$. This is true since $f(x)$ is not a square. Also $y \notin K(x)$. If $y \in K(x)$ then $K(x)=K(x,y)=F$ (same for the case $F=K(x,z)$ because $z \in K(x,y)$). We have shown that $[F:K(x)]=2$. The set also contains $2k$ elements. Therefore $l(kD) \geq 2k$. 

We know that for a sufficiently large $k$ (if $l(kD) \geq 2g-1$, $g$ genus) we have $l(kD) = \deg(kD)-g+1$ having $\deg(kD)=2k, l(kD) \geq 2k \implies 0 \leq l(kD)-\deg(kD) = -g + 1 \iff g \leq 1$.

$\qed$

\textbf{Proposition Q.5.} \textit{Proof:}
As noted by paragraph before Q.5. w.l.o.g. we can assume $f(x) = x^4+bx^2+cx+d$. Denote $bx^2+cx+d = g(x)=f(x)-x^4$. First we will prove that for both $z \in Z = \{y+x^2, y-x^2\}: [F:K(z)]=2$. 

Denote $z_1=y+x^2, z_2=y-x^2$. First we show that $F=K(x,z_i)$ for $i=1,2$. $F$ can be expressed as $K(x,y)$. $y\in K(x,z_i)$ since $z_i \pm x^2 = y$. This shows $F \subseteq K(x,z_i)$ and the converse is obvious. Also $K(x,z_i) \neq K(z_i)$ because for genus 1 it is a contradiction. If genus is 0 then $F=K(x+y)$ and it would mean $K(x+y)=K(y \pm x^2)$. 

We will find minimal polynomial $m(T)$ of $x$ over $K(z_1)$ then $\deg(m) = [F:K(z_1)]=[K(x,z_1):K(z_1)]$. $z_2 = z_1 - 2x^2$ and $z_1z_2=y^2-x^4=g(x)$. Then:
\begin{gather*}
z_1(z_1-2x^2) = g(x) = bx^2+cx+d
\end{gather*}
Then $m(T)=z_1(z_1-2T^2)-bT^2-cT-d \in K(z_1)[T]$. $\deg(m)=2$ and $m(x)=0$ and $F\neq K(z_1) \implies$ this must be a minimal polymial of $x$ over $K(z_1)$.

In a similar way we can find a minimal polynomial of $x$ over $K(z_2): m(T) = z_2(z_2+2T^2)-bT^2-cT-d \in K(z_2)[T]$.

We have shown $[F:K(z_i)]=2$.


Choose $P \leq D$ a place. We will prove that for at least one $z \in Z: v_P(z) < v_P(x)$.
\begin{gather*}
y^2 = f(x) \iff y^2 - x^4 = g(x), 0 \leq \deg(g) \leq 3 \implies\\
(y+x^2)(y-x^2)=g(x) \implies v_P(y+x^2)+v_P(y-x^2) = v_P(g(x)) = \deg(g)v_P(x)
\end{gather*}
Denote $z_1 = y+x^2, z_2=y-x^2$. Assume to contrary $v_P(z_1) \geq v_P(x)$ and $v_P(z_2) \geq v_P(x)$. We will look at all possible cases.

$\deg(g)=3$: $3v_P(x)=v_P(z_1)+v_P(z_2) \geq v_P(x)+v_P(x) \implies v_P(x) > 0$ which is a contradiction since $v_P(x)<0$.

$\deg(g)=2$: $2v_P(x)=v_P(z_1)+v_P(z_2)$. 

First consider $v_P(x)=v_P(z_1)=v_P(z_2)$ and $v_P(x)=-2$, then $2P = (x)_{-}= (z_1)_{-} = (z_2)_{-}$. Then $z_1, z_2 \in \mathcal{L}(D)$ but then also $\frac{z_1+z_2}{2} = y \in \mathcal{L}(D)$ which is a contradiction since $(y)_{-}=2(x)_{-}$.

Now let's assume $v_P(x)=-1$ and then there also exist a different place $Q$ s.t. $v_Q(x)=v_Q(z_1)=v_Q(z_2)=-1$. Since $\deg((z)_{-})=2$ then again $P+Q = (x)_{-}= (z_1)_{-} = (z_2)_{-}$ and we have the same contradiction.

We have proven that $v_P(x)=v_P(z_1)=v_P(z_2)$ is impossible. Therefore for one $z$ it must hold $v_P(z)<v_P(x)$.

$\deg(g)=1$: $v_P(x)=v_P(z_1)+v_P(z_2)$. First assume $v_P(x)=-2$. And also $v_P(z_1)=v_P(z_2)=-1$. This is impossible since $\deg((z)_{-})=2$ but for every other place $Q \neq P: 0 = v_Q(z_1)+v_Q(z_2) \implies v_Q(z_1)=-v_Q(z_2)$. If there was $Q_1: v_{Q_1}(z_1)=-1 \implies v_{Q_1}(z_2)=1$ and $Q_2: v_{Q_2}(z_2)=-1 \implies v_{Q_2}(z_1)=1$. Then for some $P_1,P_2$ places of degree 1: $(z_1)=(P_1+Q_2)-(P+Q_1), (z_2)=(P_2+Q_1)-(P+Q_2)$. Set $D'=P+Q_1+Q_2$ then $z_1,z_2 \in \mathcal{L}(D')$ and as before this means that $y \in \mathcal{L}(D')$ which is a contradiction.

Now if $v_P(x)=-1$ then either $v_P(z) < v_P(x)$ for a $z \in Z$ or w.l.o.g $v_P(z_1)=-1$ and $v_P(z_2)=0$. Then also assume first $v_Q(z_1)=-1 \implies v_Q(z_2)=0$. But since $\deg((z)_{+})=2$ there must be a place $P'$ s.t. $v_{P'}(z_2)>0$ and $v_{P'}(x)=0$ since $P'\neq P,Q$ but it must be $v_{P'}(z_1) < 0$. This again contradicts the degree of the divisor. 

If $v_Q(z_1)=0$ and $v_Q(z_2)=-1$. Then again there must be a place $P_1$ s.t. $v_{P_1}(z_1)=-1$ and $v_{P_1}(z_2)=1$ and a place $v_{P_2}(z_2)=1 \implies v_{P_2}(z_1)=-1$. This also contradicts divisor degree.

The last case is $\deg(g)=0$: $v_P(z_1)=-v_P(z_2) \implies (z_1) = -(z_2)$. First assume $v_P(z_1)=0 \implies v_P(z_2)=0$ and same for $v_Q$ ($Q$ not necessarly different from $P$). Then there exist places $P_1,P_2,Q_1,Q_2 \neq P,Q$ s.t. $(z_1)=P_1+P_2-(Q_1+Q_2), (z_2)=-(z_1)$. Set $D' = P_1+P_2+Q_1+Q_2$ then $z_1,z_2 \in \mathcal{L}(D')$ but also $y \in \mathcal{L}(D')$ which is again contradiction since $(y)_{-}=2(x)_{-}$.

If $v_P(z_1)=1 \implies v_P(z_2)=-1$. There exists another place $P'$ s.t. $v_{P'}(z_1)=1 \implies v_{P'}(z_2)=-1$. There must be again two places $Q_1,Q_2$ s.t. $(z_1)=P+P'-(Q_1+Q_2), (z_2) = -(z_1)$. Put $D' = P+P'+Q_1+Q_2$ then again $y \in \mathcal{L}(D')$ which is a contradiction.
\\

We have proven that for each place $P \leq D$ at least one $z \in Z$ must have $v_P(z)<v_P(x)$. This shows also that $(x)_{-}=P+Q$ for distinct $P,Q$. If $P=Q$ then $v_P(x)=-2 \implies v_P(z)\leq -3$ which contradicts $[F:K(z)]=2$. Since $\deg((z)_{-})=2$ and $v_P(z) < -1$ it must be that $(z)_{-}=2P, (z')_{-}=2Q$ for $z,z' \in Z$. Similar argument can be used for if $(x)_{-} = P$ where $\deg(P)=2$. It would have to be that $v_P(z) \leq -2$ which would contradict $[F:K(z)]=2$.

Since we have not distinguished $P$ and $Q$ we can say $(z_1)_{-}=2P$ and $(z_2)_{-}=2Q$.



$\qed$


\textbf{Theorem Q.6.} \textit{Proof:}
Assume genus 0. There exists $t \in F$ s.t. $(t)=P-Q$ and also $(t^{-1})=-(t)=Q-P$. Also $l(D)=\deg(D)+1 = 3$. 

$t \in \mathcal{L}(D)$ since $(t)+D = P-Q+P+Q=2P \geq 0$. Also $t^{-1} \in \mathcal{L}(D): (t^{-1})+D=Q-P+P+Q = 2Q \geq 0$. $t$ and $t^{-1}$ are linearly independent since $t \notin K$. This means $\{1,t,t^{-1}\}$ is a basis of $\mathcal{L}(D)$.

$x \in \mathcal{L}(D) \implies x = c_0+c_1t+c_2t^{-1}$ for some $c_i \in K$. This is equivalent to saying $tx = u(t), u(t) \in K[t], \deg(u)=2$.

In the same way we see $\{1,t,t^{-1},t^2,t^{-2}\}$ for a basis of $\mathcal{L}(2D)$. Again $y \in \mathcal{L}(2D): (y) + 2D = (y)_{+} - (y)_{-} + 2(x)_{-} = (y)_{+} \geq 0$. This means $t^2y = v(t)$ where $v(t) \in K[t], \deg(v)=4$.

$y^2 = f(x) \iff t^4y^2=t^4f(x)$. Substitute $yt=v(t)$ and $xt^2=u(t)$ then we have equality $v^2(t)=t^4f(\frac{u(t)}{t})$. $f$ is a polynomial of degree $4$ therefore it has up to $4$ different roots $1 \leq i \leq 4: \alpha_i \implies v^2(t)=t^4(\frac{u(t)}{t}-\alpha_1)(\frac{u(t)}{t}-\alpha_2)(\frac{u(t)}{t}-\alpha_3)(\frac{u(t)}{t}-\alpha_4)$ we can rewrite this as 
\begin{gather*}
v^2(t)=(u(t)-t\alpha_1)(u(t)-t\alpha_2)(u(t)-t\alpha_3)(u(t)-t\alpha_4)
\end{gather*}
$v^2(t)=v(t)v(t)$ is a polynomial of degree $8$, which has at most 4 different roots. Also $u(t)-t\alpha_i$ is a polynomial of degree 2. There exist at most two $\alpha \in K$ s.t. $u(t)-t\alpha$ has a root of multiplicity 2. This is because the root of a quadratic polynomial has multiplicity 2 when the discriminant D is 0. If $g(x)-\alpha x=ax^2+(b-\alpha)x+c \implies 0 = \text{D} = (b-\alpha)^2-4ac \iff \alpha = \pm 2\sqrt{ac}+b$.

Also if $i\neq j: \alpha_i \neq \alpha_j$ the polynomials $u(t)-t\alpha_i$ and $u(t)-t\alpha_j$ do not have common roots. This means that if $1 \leq i \leq 4: \alpha_i$ are all different then $v^2(t)$ has at least $6 = 2+2+1+1$ different roots. This a contradiction.

Therefore if genus is $0$ then $f$ cannot be separable. We have shown that genus of $F$ is 0 or 1, this means that for $f$ separable we must have genus 1.

Denote $w=x_2^2-f(x_1)$.
\begin{gather}
\frac{\partial w}{\partial x_1} = -f'(x_1), \frac{\partial w}{\partial x_1} = 2x_2
\end{gather}
For a singularity $\alpha = (\alpha_1, \alpha_2), \alpha_2$ must be $0$ and $\alpha_1$ must be a root of $f(x_1)$ and also of $f'(x_1)$. That is true iff $f(x_1)$ is not separable. If $f(x_1)$ is not separable then it shares a common root $\alpha_1$ with $f'(x_1)$ and this gives us singularity at $(\alpha_1, 0)$. This proves the rest of the theorem.

$\qed$

\textbf{Theorem Q.7.} \textit{Proof:}
Denote $D = P + Q$ a divisor. Due to genus being 1 $\forall k\geq 1: l(kD) = 2k$. $l(D)=2$ and that means there exists $x \notin K$ s.t. $\{1,x\}$ is a basis of $\mathcal{L}(D)$ and s.t. $(x)_{-} = P+Q$. Then $(x)^2 = 2(x) = 2(x)_{+}-2(x)_{-} \implies x^2 \in \mathcal{L}(2D),\{1,x,x^2\}$ is linearly independent in $\mathcal{L}(2D)$ but $l(2D)=4$ that means there exists $y \in \mathcal{L}(2D)\setminus \mathcal{L}(D)$ such that $\{1,x,x^2,y\}$ is a basis of $\mathcal{L}(2D)$. 

Denote $B = \{1,x,x^2,x^3,x^4,y,yx,yx^2,y^2\}$, clearly $B \subseteq \mathcal{L}(4D), l(4D)=8$ and $|B|=9 \implies 1 \leq i \leq 8: \exists a_i \in K:$
\begin{gather*}
y^2 = a_1y + a_2yx + a_3yx^2 + a_4x^4 + a_5x^3 + a_6x^2 + a_7x+ a_8
\end{gather*}

Denote $C = \{1,x,x^2,x^3,y,yx\}$. $C$ is a basis of $\mathcal{L}(3D)$, $C \cup \{yx^2,y^2\}$ is also a basis of $\mathcal{L}(4D)$ since $y^2,yx^2 \in \mathcal{L}(4D) \setminus \mathcal{L}(3D)$ because we have chosen $y$ s.t. $(y)_{-} \geq 2P$ or $(y)_{-} \geq 2Q$.

If $a_4=0$ that would be a contradiction to $C \cup \{yx^2,y^2\}$ being a basis of $\mathcal{L}(4D)$ since $y^2$ would be a linear combination of 7 elements.

Now we make a substitution $y \rightarrow y - \frac{a_1+a_2x+a_3x^2}{2}$. This gives us form:
\begin{gather*}
y^2 =b_1x^4 + b_2x^3 + b_3x^2 + b_4x+ b_5
\end{gather*}
where $b_1 = a_4+\frac{a_3^2}{4}$. If $b_1=0$ then $y^2$ would be a linear combination of elements in $\mathcal{L}(3D)$ that is a contradiction.

We know $2=\deg((x)_{-})=[F:K(x)]$. We have $K(x,y) \subseteq F$. We want to show $F=K(x,y)$.
\begin{gather*}
2=[F:K(x)]=[F:K(x,y)][K(x,y):K(x)]
\end{gather*}
If $[F:K(x,y)] = 1$ we are done. Assume $[F:K(x,y)]=2$ which means $[K(x,y):K(x)]=1$ i.e. $K(x,y)=K(x)$. We know that $p(T)=T^2-g(x) \in K(x)[T]$ where $g(x)=b_1x^4 + b_2x^3 + b_3x^2 + b_4x+ b_5$ has $y$ as a root in $K(x,y)$. Since $[K(x,y):K(x)]=1$ there must exist a polynomial $T-h(x) \in K(x)[T]$ s.t. $T-h(x)|p(T)$ and it must be $\deg(h)=2$. This implies $y=ax^2+bx+c$ for some $a,b,c \in K$. This is a contradiction with selection of $y$ since we are selecting it to $\{1,x,x^2,y\}$ be a basis of $\mathcal{L}(2D)$.

$\qed$

\textbf{Theorem Q.8.} \textit{Proof:}
Denote $f(x) = g(x^2)$. $F$ is EFF therefore genus is 1 and there exists a place of degree $1$.

If $g(x)$ has a multiple root $\alpha$, then $f(x)=g(x^2)$ has also a multiple root because $g(x)=(x-\alpha)^2 \implies g(x^2) = (x-\sqrt{\alpha})^2(x+\sqrt{\alpha})^2$. Set $z = \frac{y}{x-\sqrt{\alpha}}$. Then $F$ is given by $z^2=(x+\sqrt{\alpha})^2$. This means (using same technique as in Q.2) that $F = K(x+z)$ which means $F$ has genus 0, a contradiction.

From now on we can assume $g(x)$ has 2 distinct roots. If $f(x)=g(x^2)$ would have a multiple root then it's genus would not be $1$ by Q.6. So we can assume $g(x^2)$ separable. 

First we will prove the second part of the theorem. We have shown that $g(x^2)$ must be separable. Therefore by Q.5 we have places of degree 1 ($P\neq Q$), $(x)_{-} = P+Q$ and $(y+x^2)_{-}=2Q, (y-x^2)_{-} = 2P$.
\begin{gather*}
y^2=g(x^2) = x^4+2bx^2+c \iff y^2-(x^4-2bx^2-b^2) = c - b^2 \implies \\
(y-(x^2+b))(y+(x^2+b))=c-b^2
\end{gather*}

Since $g(x^2)$ is separable $g(x)$ must have simple roots. If $g(x)$ has a multiple root then it's discriminant is 0 and that happens iff $c-b^2=0$. So we know $0 \neq c-b^2 \in K$.

\begin{gather*}
0=v_P(c-b^2)=v_P(y-(x^2+b)) + v_P(y+(x^2+b))\\
v_P(y-(x^2+b)) = v_P(y-x^2) = -2 \implies v_P(y+(x^2+b)) = 2
\end{gather*}

Similarly we can show $v_Q(y-(x^2+b)) = 2$ and $v_Q(y+x^2+b)=-2$. 

Since $\deg((y+x^2+b)_{+}) = \deg((y+x^2+b)_{-}) = \deg((y+x^2)_{-}) = 2 \implies \text{div}(y+x^2+b) = 2P - 2Q$ and similarly $\text{div}(y+x^2+b) = 2Q - 2P$.

We have proven the last part of the theorem. Now let's prove the equivalence.

As we have shown before. We can assume $g(x^2)$ separable and then we have involution $P-Q$ as shown above since $2P-2Q = (t)$ for $t \in F$. We only have to show that $P-Q \neq (t)$ for some $t \in F$. We know that if $P-Q \in \text{Princ}(F/K) \implies P=Q$ so that would be a contradiction. 

Now we assume we have involution. This mean we have $P \neq Q$ places of degree one s.t. $2[P-Q]=(t), t \in F$ and $P-Q \neq (s), s \in F$. Using Q.7 we know $F$ is given by $y^2 = f(x), \deg(f)=4$ and $f$ monic where $x,y \in F, (x)_{-} = P+Q$. We can also assume $f$ separable since $F$ is of genus 1 using Q.6. If $f$ has no simple root then similarly it can be shown $F$ has genus 0. We can also take $f$ s.t. $\deg(f(x)-x^4)=2$ due to the paragraph before Q.5.

By Q.5 we also have $(y-x^2)_{-} = 2P$ and $(y+x^2)_{-} = 2Q$. We know that $t \in \mathcal{L}(2P), t^{-1} \in \mathcal{L}(2Q)$. Since $t \notin K$ $\{1,t\}$ forms a basis of $\mathcal{L}(2P)$ and $\{1,t^{-1}\}$ forms a basis of $\mathcal{L}(2Q)$ therefore there exist $p_1, p_2, q_1, q_2 \in K$ s.t. $y-x^2 = p_1+p_2t, y+x^2=q_1+q_2t^{-1}$. We know that $f(x)=x^4+bx^2+cx+d, b,c,d \in K$. We want to show that $c=0$. 

We have $y^2-x^4=(y-x^2)(y+x^2)=bx^2+cx+d \iff (p_1+p_2t)(q_1+q_2t^{-1})-d = bx^2+cx$. If we use new names for the coefficients on the left side $(s_1,s_2,s_3 \in K, s_1,s_2 \neq 0)$
 we get:
\begin{gather*}
s_1t+s_2t^{-1}+s_3 = bx^2+cx
\end{gather*}
We know $v_P(t)=2$ and $v_P(t^{-1})=-2$. Assume $s_3 \neq 0$. Then $v_P(s_2t^{-1}+s_3) = -2$ since $v_P(s_3) = 0$. Also this means that $v_P(s_1t+s_2t^{-1}+s_3)=-2$. If $s_3=0$ we get the same result. We now have:
\begin{gather*}
-2=v_P(bx^2+cx) = v_P(x)+v_P(bx+c)
\end{gather*}
Since $v_P(x)=-1$ we have $v_P(bx+c) = -1$.

Now we have $s_1t+s_2t^{-1}+s_3 = bx^2+cx \iff cx = s_1t+s_2t^{-1}+s_3-bx^2$. Assume $c \neq 0$. If $b=0$ then $x \in K(t)$ and since $y+x^2 \in K(t)$ we have $y \in K(t) \implies F=K(t)$ which is a contradiction with genus being 1. If $b \neq 0$ then $x \in K(t,x^2)$ and also since $(y-x^2)-(y+x^2) \in K(t) \implies x^2 \in K(t) \implies x \in K(t)$ and also $y \in K(t)$ which is again a contradiction with $F$ being genus 1.

Therefore it must be $c=0$.

$\qed$

\textbf{Proposition G.2} \textit{Proof:}
Since $F/K$ is given by $y^2 = g(x^2)$ we know that $F=K(x,y)$.

Clearly $K(\tilde{x},\tilde{y}) \subseteq K(x,y)$. The other inclusion is also clear since $x = \tilde{x}^{-1}\tilde{y}$ and $y = 2\tilde{x}-x^2-b$. 

Now we need to show that $-\tilde{y}^2+\tilde{x}^3-b\tilde{x}^2+\frac{b^2-c}{4}\tilde{x} = 0$ in $F$.

\begin{gather*}
\tilde{y}^2 = \tilde{x}^3-b\tilde{x}^2+\frac{b^2-c}{4}\tilde{x}\\
\text{Substitute $\tilde{x}, \tilde{y}$:}\\
\frac{x^2u^2}{4} = \frac{u^3}{8} - \frac{bu^2}{4} + \frac{b^2-c}{8}u\\
2x^2u^2 = u^3 - 2bu^2 + (b^2-c)u \\
\text{Divide by $u \neq 0$ in $F$:}\\
2x^2u = u^2 - 2bu + (b^2-c)\\
\text{Substitute $u=y+x^2+b$:}\\
2x^2(y+x^2+b) = (y^2+x^4+2yx^2+2by+2bx^2+b^2)-2b(y+x^2+b)+b^2-c\\
\iff\\
y^2-x^4-2bx^2-c = y^2 - g(x^2)= 0
\end{gather*}
We have shown that $\tilde{y}^2 = \tilde{x}^3-b\tilde{x}^2+\frac{b^2-c}{4}\tilde{x}$ in $F$ since $y^2=g(x^2)$ in $F$ by definition.

The only thing left is to show that $y^2-x^3+bx^2+\frac{b^2-c}{4}x$ is irreducibile in $K[x,y]$. Using Einsenstein criterion with $x \in K[x]$ as a prime element we get that this polynomial is indeed irreducibile in $(K[x])[y]$.

$\qed$

\textbf{Theorem G.3} \textit{Proof:}
Assume we have $(a_2, a_4) \in K \times K$ s.t. $a_4 \neq 0 \neq a_2^2-4a_4$. Set $b = -a_2$ and $c = -4a_4+a_2^2$. If $c = 0$ this would mean that $4a_4=a_2^2$ which is a contradiction with $a_2^2-4a_4 \neq 0$. If $b^2-c=0 \iff c=b^2$ then this is would imply $-4a_4+a_2^2 = a_2^2 \iff a_4 = 0$ which is again a contradiction.

On the other hand assume we have $(b,c) \in K \times K$ s.t. $c \neq 0 \neq b^2-c$. Set $a_2 = -b, a_4 = \frac{b^2-c}{4}$. If $a_4 = 0$ this would imply $b^2-c = 0$ which is contradiction. If $a_2^2-4a_4 =0 \iff b^2=b^2-c \iff c = 0$ which is a contradiction.

We have therefore proven that $G.1$ holds.

Denote the rational map $C_1 \rightarrow C_2$ as $\psi$ and the map $C_2 \rightarrow C_1$ as $\phi$. Also $f(x_1,x_2) = x_2^2-x_1^3-a_2x_1^2-a_4x_1, C_1 = V_{f}, g(x_1,x_2) = x_2^2-x_1^4-2bx_1^2-c, C_2 = V_g$.

Now we need to show that $\psi$ and $\phi$ are actually $K$-rational maps. We will use lemma $R.6$ from the lecture. 

First we know that $\phi$ is a rational map thanks to proposition $G.2$. By proposition $G.2$ (since $a_2 = -b, a_4 = \frac{b^2-c}{4}$) we have shown that $(\rho_1,\rho_2) = 0 \in K(C_2)$ where $(\rho_1,\rho_2) = \left(\frac{x_2+x_1^2+b}{2} + (g), \frac{x_1(x_2+x_1^2+b)}{2} + (g)\right)$. By lemma $R.6$ this means that $\phi$ is a $K$-rational map $C_2 \rightarrow C_1$.

To show that $\psi$ is a $K$-rational map $C_1 \rightarrow C_2$ we need to prove that $g(\rho_1, \rho_2) = 0 \in K(C_1)$ where $(\rho_1, \rho_2) = \left( \frac{x_2}{x_1} + (f), \frac{x_1^2-a_4}{x_1} + (f)\right)$.

\begin{gather*}
g(\rho_1, \rho_2) = \frac{(x_1^2-a_4)^2}{x_1^2} - \frac{x_2^4}{x_1^4} - 2b\frac{x_2^2}{x_1^2} - c\\
g(\rho_1, \rho_2) = 0 \iff x_1^4g(\rho_1, \rho_2) = 0 \text{ and using $b=-a_2, c = a_2^2-4a_4$}\implies\\
x_1^4g(\rho_1, \rho_2) = -x_2^4 + 2a_2x_1^2x_2^2+x_1^6+2a_4x_1^4-a_2^2x_1^4+a_4^2x_1^2\\
\text{Substitution $x_1^6 = (x_2^2-a_2x_1^2-a_4x_1)^2$:}
x_1^4g(\rho_1, \rho_2) = -2a_4x_1x_2^2+2a_4x_1^4+2a_2a_4x_1^3+2a_4^2x_1^2 = \\
2a_4x_1(-x_2^2+x_1^3+a_2x_1^2+a_4x_1) = 0 \in K(C_1)
\end{gather*}
This proves that $\psi$ is a $K$-rational map $C_1 \rightarrow C_2$.


By definition of birational equivalence we need to show that $\phi \circ \psi = id_{C_1}$ and $\psi \circ \phi = id_{C_2}$.

Let's start with $id_{C_1}$, we want to show that $\phi \circ \psi$ can be represented as $(x_1, x_2) \in K(C_1)$. First coordinate:
\begin{gather*}
\frac{\frac{x_1^2-a_4}{x_1}+\frac{x_2^2}{x_1^2}+b}{2} = \frac{x_1^3-a_4x_1+x_2^2+x_1^2b}{2x_1^2}\\
\text{Substitution for $x_2^2$ in $K(C_1)$ and using $b=-a_2$:}\\
\frac{2x_1^3}{2x_1^2} = x_1
\end{gather*}
Second coordinate (again using $b=-a_2$):
\begin{gather*}
\frac{\frac{x_2}{x_1}\left(\frac{x_1^2-a_4}{x_1} + \frac{x_2^2}{x_1^2} - a_2 \right)}{2} = \frac{x_2(x_1^3-a_4x_1+x_2^2-a_2x_1^2)}{2x_1^3}\\
\text{Substitution for $x_2^2$ again:}\\
\frac{x_2(2x_1^3)}{2x_1^3} = x_2
\end{gather*}
We have shown that $\phi \circ \psi = id_{C_1}$ i.e. 

Now for $\psi \circ \phi$. First coordinate:
\begin{gather*}
\frac{\frac{x_1(x_2+x_1^2+b)}{2}}{\frac{x_2+x_1^2+b}{2}} = \frac{x_1(x_2+x_1^2+b)}{x_2+x_1^2+b} = x_1
\end{gather*}
Second coordinate:
\begin{gather*}
\frac{\left(\frac{x_2+x_1^2+b}{2}\right)^2 - a_4}{\frac{x_2+x_1^2+b}{2}} = \frac{\frac{x_2^2+x_1^4+2x_1^2x_2+2bx_2+2bx_1^2+b^2-4a_4}{4}}{\frac{x_2+x_1^2+b}{2}}\\
\text{Using substitution for $x_1^4$ in $K(C_2)$ and $-4a_4 = -b^2+c$:}\\
\frac{2x_2^2+2x_1^2x_2+2bx_2}{2x_2+2x_1^2+2b} = x_2
\end{gather*}
We have proved that both compositions of those rational maps are identitities therefore $C_1$ and $C_2$ are birationally equivalent.

$\qed$

\textbf{Theorem G.4} \textit{Proof:}
Assume we have $(a_2, \gamma) \in K \times K$ s.t. $\gamma^2 \neq 0 \neq a_2^2-4\gamma^2$. Set $B = \gamma^{-1}, A = a_2\gamma^{-1}$. Since $\gamma \neq 0$ then also $\gamma^{-1} = B$ cannot be 
0. If $A^2 = 4$ then $a_2^2\gamma^{-2} = 4 \iff a^2 - 4\gamma \neq 0$ which is a contradiction.

On the other hand if we have $(A,B) \in K \times K$ s.t. $B(A^2-4)\neq 0$ then set $\gamma = B^{-1}$ and $a_2 = AB^{-1}$. If $\gamma^2 = 0$ this would imply $\gamma = 0$ which is a contradiction with $B \neq 0$. If $a_2^2-4\gamma^2 =0 \iff a_2^2 = 4\gamma^2$ this would imply $\frac{A^2}{B^2} = 4B^{-2} \iff A^2 = 4$ which is again contradiction.

We have therefore proven that $G.2$ holds.

As in $G.3$ denote $f(x_1,x_2) = x_2^2-x_1^3-a_2x_1^2-a_4x_1, C_1 = V_f$ and $g(x_1,x_2) = Bx_2^2-x_1^3-Ax_1^2-x_1, C_2 = V_g$. Denote the rational map $C_1 \rightarrow C_2$ as $\psi$ and the rational map $C_2 \rightarrow C_1$ as $\phi$.

First we show that $\psi$ is a $K$-rational map $C_1 \rightarrow C_2$ using $R.6$ as in $G.3$. We want to show $g(\rho_1, \rho_2) = 0 \in K(C_1)$ where $(\rho_1, \rho_2) = \left(\gamma^{-1}x_1 + (f), \gamma^{-1}x_2 + (f) \right)$. We know that $\gamma^{-1} = \frac{1}{\sqrt{a_4}}$ and $a_4 \neq 0, B = \frac{1}{\sqrt{a_4}}, A = \frac{a_2}{\sqrt{a_4}}$.
\begin{gather*}
g(\rho_1, \rho_2) = \frac{1}{\sqrt{a_4}}\left( \frac{x_2}{\sqrt{a_4}} \right)^2 - \frac{x_1^3}{\sqrt{a_4}^3} - \frac{a_2}{\sqrt{a_4}}\frac{x_1^2}{\sqrt{a_4}^2} - \frac{x_1}{\sqrt{a_4}} = \frac{x_2^2-x_1^3-a_2x_1^2-a_4x_1}{a_4\sqrt{a_4}} = 0
\end{gather*} 

Now we want to show $f(\rho_1, \rho_2) = 0 \in K(C_2)$ where $(\rho_1, \rho_2) = \left( B^{-1}x_1 + (g), B^{-1}x_2+(g) \right)$. Similarly $a_4 = \frac{1}{B^2}$ since $B \neq 0$ and $a_2 = \frac{A}{B}$:
\begin{gather*}
f(\rho_1, \rho_2) = \frac{x_2^2}{B^2} - \frac{x_1^3}{B^3} - \frac{A}{B}\frac{x_1^2}{B^2}-\frac{1}{B^2}\frac{x_1}{B_1} = \frac{Bx_2^2-x_1^3-Ax_1^2-x_1}{B^3} = 0
\end{gather*}

So $\psi$ and $\phi$ are both $K$-rational maps. Now as in $G.3$ we show birational equivalence by proving $\phi \circ \psi = id_{C_1}$ and $\psi \circ \psi = id_{C_2}$ which is trivial in this case.
\begin{gather*}
\phi \circ \psi = (B^{-1}(Bx_1), B^{-1}(Bx_2)) = (x_1, x_2)\\
\psi \circ \phi = (\gamma^{-1}(\gamma x_1), \gamma^{-1}(\gamma x_2)) = (x_1, x_2)
\end{gather*}

$\qed$

\textbf{Proposition G.5} \textit{Proof:}
Assume we have $(a,d) \in K^* \times K^*, a \neq d$. Set $B =  \frac{4}{a-d}, A = \frac{2(a+d)}{a-d}$. $B \neq 0$ and if $A = 2$ then this is a contradiction with $d \neq 0$ and if $A = -2$ then it is contradiction with $a \neq 0$. 

On the other hand assume $(A,B) \in K \times K$ s.t. $B \neq 0$ and $A^2 \neq 4$. Set $a = \frac{A+2}{B}$ and $d = \frac{A-2}{B}$. If $a = 0$ or $d = 0$ this is a contradiction with $A \neq \pm 2 \iff A^2 \neq 4$. If $a=d \iff 2=-2$ which is a contradiction. We have shown that $G.3$ holds.

Denote the rational map $C_1 \rightarrow C_2$ as $\psi$ and the rational map $C_2 \rightarrow C_1$ as $\phi$. First let's prove that $\psi$ is a rational map.

Theorem $G.4$ gives us a $K$-rational map $\Psi_1: C_1 \rightarrow C'$ s.t. $\Psi_1(x_1, x_2) = (B^{-1}x_1, B^{-1}x_2)$ where $C' = V_f$ where $f(x_1, x_2) = x_2^2-x_1^3-a_2x_1^2-a_4x_1$ and $a_2 = \frac{A}{B}$ and $a_4 = \frac{1}{B^2}$. Theorem $G.3$ gives us a $K$-rational map $\Psi_2: C' \rightarrow C_2$ where $\Psi_2(x_1, x_2) = \left(\frac{x_2}{x_1}, \frac{x_1^2-a_4}{x_1} \right)$. We can use $G.3$ because due to the paragraph before $G.5$ we know that $C_2$ in $G.5$ and $C_2$ in $G.3$ are equal.

By composition of these 2 $K$-rational maps we get $\Psi_2 \circ \Psi_1 = \psi$. Since $\Psi_1, \Psi_2$ are $K$-rational maps of finite degree (because every $\rho$ we used in the proofs were transcendental over $K$) we get by Theorem $R.9$ from the lecture that $\psi$ is also a $K$-rational map $C_1 \rightarrow C_2$.

Using similar process theorem $G.3$ gives us a $K$-rational map $\Phi_1: C_2 \rightarrow C'$ s.t. $\Phi_1(x_1, x_2) = \left(\frac{x_2+x_1^2+b}{2}, \frac{x_1(x_2+x_1^2+b)}{2} \right)$ where $C' = V_g$ where $g(x_1,x_2) = x_2^2-x_1^3-a_2x_1^2-a_4x_1$ where $a_2 = -b, a_4 = \frac{b^2-c}{4}$ where (using paragraph above $G.5$) $b = \frac{-a-d}{2}, c = ad$ so in total $a_2 = \frac{a+d}{2}$ and $a_4 = \frac{(a-d)^2}{16}$. Theorem $G.4$ gives us a $K$-rational map $\Phi_2: C' \rightarrow C_1$ s.t. $\Phi_2(x_1, x_2) = (Bx_1, Bx_2)$ where in theorem $G.4$ we have $A = \frac{a_2}{\sqrt{a_4}}$ and $B = \frac{1}{\sqrt{a_4}}$. If we resubstitute we get $A = \frac{2(a+d)}{a-d}$ and $B = \frac{4}{a-d}$.

Again the composition $\Phi_2 \circ \Phi_1 = \phi$. Since $\Phi_2, \Phi_1$ are $K$-rational maps of finite degree then $\phi$ is a $K$-rational map $C_2 \rightarrow C_1$.

Now we need to show $\phi \circ \psi = id_{C_1}$ and $\psi \circ \phi = id_{C_2}$.
\begin{gather*}
\phi \circ \psi = (\Phi_2 \circ \Phi_1) \circ (\Psi_2 \circ \Psi_1) = \Phi_2 \circ (\Phi_1 \circ \Psi_2) \circ \Psi_1 = \Phi_2 \circ id_{C'} \circ \Psi_1 =\\
\Phi_2 \circ \Psi_1 = id_{C_1}
\end{gather*}
We have used Theorem $G.3$ which states $\Psi_2 \circ \Psi_1 = id_{C'}, \Phi_2 \circ \Psi_1 = id_{C_1}$. Similarly due to Theorem $G.5$:
\begin{gather*}
\psi \circ \phi = (\Psi_2 \circ \Psi_1) \circ (\Phi_2 \circ \Phi_1) = \Psi_2 \circ id_{C'} \circ \Phi_1 = \Psi_2 \circ \Phi_1 = id_{C_2}
\end{gather*}
We have shown that $C_1$ and $C_2$ are birationally equivalent.

$\qed$

\textbf{Lemma W.1} \textit{Proof:}
Denote $f(x_1,x_2) = a_1x_1^2+a_2x_2^2-1-dx_1^2x_2^2 \in K[x_1, x_2]$. First we will prove $f$ absolutely irreducible $\implies d \neq a_1a_2$ and at least one $a_i \neq 0$.

Assume $d = a_1a_2$. Then $f(x_1,x_2) = a_1x_1^2+a_2x_2^2-1-a_1a_2x_1^2x_2^2 = (1-a_1x_1^2)(-1+a_2x_2^2)$. If $a_1 = 0 = a_2$ then $f(x_1,x_2) = -1-dx_1^2x_2^2 = (\sqrt{-d}x_1x_2-1)(\sqrt{-d}x_1x_2+1) \in \bar{K}[x_1, x_2]$. This proves the implication.

On the other hand assume $d \neq a_1a_2$, at least one $a_i \neq 0$ and $f(x_1,x_2) = u(x_1,x_2)v(x_1,x_2)$ where $u,v \in \bar{K}[x_1,x_2]$. We have $2 = \deg_{x_1}(f) = \deg_{x_1}(u)+\deg_{x_1}(v)$ and $2 = \deg_{x_2}(f) = \deg_{x_2}(u)+\deg_{x_2}(v)$. We can assume that $u,v \notin \bar{K}$ because then it would contradict $f$ reducible in $\bar{K}[x_1,x_2]$.

W.l.o.g. we have 4 possibilities:
\begin{enumerate}
    \item $\deg_{x_1}(u) = 0, \deg_{x_2}(u) = 1, \deg_{x_1}(v) = 2, \deg_{x_2}(v)=1$
    \item $\deg_{x_1}(u) = 0, \deg_{x_2}(u) = 2, \deg_{x_1}(v) = 2, \deg_{x_2}(v)=0$
    \item $\deg_{x_1}(u) = 1, \deg_{x_2}(u) = 0, \deg_{x_1}(v) = 1, \deg_{x_2}(v)=2$
    \item $\deg_{x_1}(u) = 1, \deg_{x_2}(u) = 1, \deg_{x_1}(v) = 1, \deg_{x_2}(v)=1$
\end{enumerate}
For each case we will find a contradiction.

Case 1: $u = ax_2+b \in \bar{K}[x_2], v = cx_1^2+ex_1^2x_2+fx_2+gx_1+h \in \bar{K}[x_1,x_2]$. If we compare the coefficients $f = uv$ we ge conditions: $bh=-1, bf+ah=0, af=a_2, bg=0, ag=0, bc=a_1, ac+be=0, ae=-d$.
$bh = -1 \implies b \neq 0 \neq h, ac+be= 0 \implies e = \frac{-ac}{b}, bc=a_1 \implies c = \frac{a_1}{b} \implies e = \frac{-a_1a}{b^2} \implies ae = \frac{-a_1a^2}{b}$. If $a_2 =0 \implies a=0$ or $f = 0$ if $a=0 \implies -d = 0$ a contradiction. If $f = 0 \implies bf+ah = ah = 0$ and since $h \neq 0 \implies a = 0$ again. So we can assume $a_2 \neq 0 \implies f \neq 0$. Now $ae = \frac{-a_1a^2}{b^2} = \frac{-a_1a^2f}{b^2f} = \frac{-a_1a_2a}{b^2f}$. If $\frac{a}{b^2f} = 1$ we have $d = a_1a_2$ a contradiction. $\frac{a}{b^2f} = \frac{ah}{bbhf} = \frac{-bf}{-bf} = 1$.

Case 2: $u = ax_2^2+bx_2+c \in \bar{K}[x_2], v = ex_1^2+fx_1+g \in \bar{K}[x_1]$. Again we get conditions: $cg=-1, bg=0, ag=a_2, cf=0,bf=0, af=0, ce=a_1, be=0, ae=-d$. We know $c \neq 0 \neq g$ and this gives us $a = \frac{a_2}{g}, e = \frac{a_1}{c} \implies ae = \frac{a_1a_2}{cg} = -a_1a_2 = -d$ a contradiction.

Case 3: $u = ax_1+b \in \bar{K}[x_1], v = cx_2^2+ex_2^2x_1+fx_2+gx_1+h \in \bar{K}[x_1,x_2]$. We get: $bh=-1, bf=0, bc=a_2, bg+ah=0, af=0, ac+be=0, ag=a_1, ae=-d$. We have $b \neq 0 \neq h$. If $a_1 = 0$ this would imply $a = 0$ or $g = 0$. If $a = 0 \implies -d = 0$ a contradiction. If $g = 0 \implies ah = 0 \implies a = 0$ again a contradiction. Therefore we have $a = \frac{a_1}{g}$. Also $bc=a_2 \iff c = \frac{a_2}{b}$ and $e = \frac{-ac}{b} \implies e = \frac{-aa_2}{b^2}$. Together $ae = \frac{-aa_2a_1}{b^2g}$. Again we want to show $\frac{a}{b^2g} = 1$ which is true since $\frac{a}{b^2g}=\frac{a}{b(-ah)} = \frac{a}{-a(bh)} = 1$.

Case 4: We know $\deg_{x_1}(u)=1=\deg_{x_1}(v)$. Consider $f \in (K[x_2])[x_1] \iff f = x_1^2(a_1-dx_2^2)+(a_2x_2^2-1)$. Since $\deg_{x_1}(u)=1=\deg_{x_1}(v)$ means that $f = (a'x_1+b')(c'x_1+d') \in (K[x_2])[x_1]$ i.e. $f$ has roots $\frac{b'}{a'}, \frac{d'}{c'} \in \bar{K}(x_2)$. 

Since $f$ is a quadratic polynomial it must be that the discriminant of $f: D= -4(a_1-dx_2^2)(a_2x_2^2-1)$ is a square in $\bar{K}(x_2)$ which is equivalent to saying it is a square in $\bar{K}[x_2]$ since $D \in \bar{K}[x_2]$. This means that $a_1-dx_2^2$ must have a double root and same for $a_2x_2^2-1$ or they have a common root. 

If $a_1-dx_2^2$ has a double root then its discriminant is 0 $\iff a_1d = 0$ similarly for $a_2x_2^2-1$ it must be that $a_2 = 0$. If $a_1 \neq 0 \neq a_2$ we have a contradiction. If one $a_i=0$ we have also a contradiction since $d \neq 0$. 

The only possibility left it that they have a common root. $\pm\frac{\sqrt{da_1}}{d}$ are the roots of the first polynomial and $\pm\frac{\sqrt{a_2}}{a_2}$ are the roots of the other polynomial. We know that at least one $a_i$ is non zero. If $a_1 = 0, a_2 \neq 0$ then obviously they don't have common roots and same goes for $a_2 = 0, a_1 \neq 0$. Now we can assume $a_i$ are both non zero and in that case if we want a common root we get a requirement that $d=a_1a_2$ which is a contradiction.

$\qed$


\textbf{Proposition W.2} \textit{Proof:}
By lemma $Q.1$ we know that $f_1$ is absolutely irreducible since $a\neq d$ and by lemma $W.1$ we know that $f_2$ is absolutely irreducible since $a \neq d$. 

Denote the $K$-rational map from $C_1 \rightarrow C_2$ as $\psi$ and the $K$-rational map from $C_2 \rightarrow C_1$ as $\phi$. First we show $\psi$ is a $K$-rational map $C_1 \rightarrow C_2$. We want to show $f_2(\rho_1, \rho_2) = 0 \in K(C_1)$ where $(\rho_1, \rho_2) = \left( \frac{1}{x_1} + (f_1), \frac{x_2}{x_1^2-d} + (f_1) \right)$.
\begin{gather*}
f_2(\rho_1, \rho_2) = a\frac{1}{x_1^2} + \frac{x_2^2}{(x_1^2-d)^2} - 1 -d\frac{1}{x_1^2}\frac{x_2^2}{(x_1^2-d)^2}\\
(x_1^2-d)^2x_1^2f_2(\rho_1, \rho_2) = a(x_1^2-d)^2+x_1^2x_2^2-x_1^2(x_1^2-d)^2-dx_2^2 = \\
x_1^2x_2^2-dx_2^2-x_1^6+2dx_1^4+ax_1^4-d^2x_1^2-2adx_1^2+ad^2\\
\text{Substitute for $x_2^2 = x_1^4-dx_1^2-ax_1^2+ad$:}\\
-dx_2^2+dx_1^4-d^2x_1^2-adx_1^2-ad^2\\
\text{Substitute for $x_1^4 = x_2^2+dx_1^2+ax_1^2-ad$:}\\
0 \implies f_2(\rho_1, \rho_2) = 0
\end{gather*}
Now we do the same for $\phi$. We want to show $f_1(\rho_1, \rho_2) = 0 \in K(C_2)$ where $(\rho_1, \rho_2) = \left( \frac{1}{x_1} + (f_2), \frac{x_2(1-dx_1^2)}{x_1^4} + (f_2) \right)$.
\begin{gather*}
f_1(\rho_1, \rho_2) = \frac{x_2^2(1-dx_1^2)^2}{x_1^4} - \frac{1}{x_1^4} + \frac{(d+a)}{x_1^2}-ad\\
x_1^4f_1(\rho_1, \rho_2) = x_2^2(1-dx_1^2)^2-1+dx_1^2+ax_1^2-adx_1^4 = x_2^2(1-dx_1^2)^2 - (1-ax_1^2)(1-dx_1^2)\\
\text{As mentioned before $W.2$ in $K(C_2)$ we have $x_2^2(1-dx_1^2)^2 = (1-ax_1^2)(1-dx_1^2)$}\\
\implies f_1(\rho_1, \rho_2) = 0
\end{gather*}
Now we want to prove $\phi \circ \psi = id_{C_1}$ and $\psi \circ \phi = id_{C_2}$. First coordinate is trivially $x_1$ in both cases. The second coordinate for $\phi \circ \psi$:
\begin{gather*}
\frac{\frac{x_2}{x_1^2-d}\left(1- d\frac{1}{x_1^2} \right)}{\frac{1}{x_1^2}} = \frac{x_1^2x_2(1-\frac{d}{x_1^2})}{x_1^2-d} = \frac{x_2(x_1^2-d)}{x_1^2-d} = x_2
\end{gather*}
And for $\psi \circ \phi$:
\begin{gather*}
\frac{\frac{x_2(1-dx_1^2)}{x_1^2}}{\frac{1}{x_1^2}-d} = \frac{x_2(1-dx_1^2)}{1-dx_1^2} = x_2
\end{gather*}
This proves that $C_1$ and $C_2$ are birationally equivalent. 

Since $f_1, f_2$ are absolutely irreducible (which implies $f_1,f_2$ irreducible in $K[x_1,x_2]$)a nd $C_1,C_2$ are birationally equivalent we can use corollary $R.10$ from the lecture and we have that $K(C_1) \cong K(C_2)$.

If $a=0$ then $f_1(x_1,x_2) = x_2^2-f(x_1)$ where $f(x_1) = x_1^2(x_1^2-d)$. $0$ is a multiple root of $f$ therefore $f$ is not separable. Theorem $Q.6$ states that in this case $K(C_1)/K$ has genus 0.

If $a\neq 0$ then $f$ is separable since its roots are $\pm \sqrt{a}, \pm \sqrt{d}$ which are distinct since also $d \neq 0$. By $Q.6$ again the genus is 1. Since $K(C_1)/K$ and $K(C_2)/K$ are isomorphic their genera coincide.

$\qed$

\textbf{Theorem W.3} \textit{Proof:}
First assume $(a,d) \in K^* \times K^*, a \neq d$. Set $B = \frac{4}{a-d}, A = 2+\frac{4d}{a-d}$. By definition $B \neq 0$. If $A = 2 \implies d = 0$ a contradiction. If $A = -2 \implies a = 0$ again a contradiction. On the other hand assume $(A,B) \in K \times K$ s.t. $B \neq 0$ and $A^2 \neq 4$. Set $d = \frac{A-2}{B}, a = \frac{A+2}{B}$. If $a=0$ or $d=0$ contradicts $A\neq \pm2$. If $a=d$ then we have also a contradiction. This proves $(W.2)$.

Denote the $K$-rational map $C_1 \rightarrow C_2$ as $\psi$ and the $K$-rational map $C_2 \rightarrow C_1$ as $\phi$. We will use similar steps as in the proof of $G.5$.

First assume we have $(A,B) \in K\times K, B \neq 0, A^2 \neq 4$. By applying $G.5$ we get a $K$-rational map $\Psi_1: C_1 \rightarrow C'$ s.t. $\Psi_1(x_1,x_2) = \left( \frac{x_2}{x_1}, \frac{x_1^2-1}{Bx_1}\right)$ where $C' = V_g, g(x_1,x_2) = x_2^2-(x_1-a)(x_1-d)$ ($a,d$ given by $W.2$). Also by using Proposition $W.2$ we have another $K$-rational map $\Psi_2: C' \rightarrow C_2$ s.t. $\Psi_2(x_1,x_2) = \left( \frac{1}{x_1}, \frac{x_2}{x_1^2-d} \right)$. $\Psi_1$ and $\Psi_2$ are $K$-rational maps of finite degree so their composition is also a $K$-rational map of finite degree. Now we will check that $\psi = \Psi_2 \circ \Psi_1$. Using $d = \frac{A-2}{B}$ we check again the coordinates of the composition of maps. First coordinate:
\begin{gather*}
\frac{1}{\frac{x_2}{x_1}} = \frac{x_1}{x_2}
\end{gather*}
And the second:
\begin{gather*}
\frac{\frac{x_1^2-1}{Bx_1}}{\frac{x_2^2}{x_1^2}-d} = \frac{\frac{x_1^2-1}{Bx_1}}{\frac{x_2^2}{x_1^2}-\frac{A-2}{B}} = \frac{x_1(x_1^2-1)}{Bx_2^2-x_1^2(A-2)}\\
\text{Substitute $Bx_2^2 = x_1^3+Ax_1^2+x_1$:}\\
\frac{x_1(x_1-1)(x_1+1)}{x_1^3+2x_1^2+x_1} = \frac{(x_1-1)(x_1+1)}{(x_1+1)^2} = \frac{x_1-1}{x_1+1}
\end{gather*}
We have shown $\Psi_2 \circ \Psi_1 = \psi$. Therefore $\psi$ is a $K$-rational map $C_1 \rightarrow C_2$.

Now assume we have $(a,d) \in K^* \times K^*, a\neq d$. By Proposition $W.2$ we have a $K$-rational map $\Phi_1: C_2 \rightarrow C'$ s.t. $\Phi_1(x_1,x_2) = \left(\frac{1}{x_1}, \frac{x_2(1-dx_1^2)}{x_1^2} \right)$ where $C' = V_g, g(x_1,x_2) = x_2^2-(x_1-a)(x_1-d)$. Using theorem $G.5$ we also have a $K$-rational map $\Phi_2: C' \rightarrow C_1$ s.t. $\Phi_2(x_1,x_2) = \left( \frac{2(x_2+x_1^2)-(a+d)}{a-d}, x_1\frac{2(x_2+x_1^2)-(a+d)}{a-d}\right)$. $\Phi_1, \Phi_2$ are $K$-rational maps of finite degree and therefore their composition is also a $K$-rational map of finite degree.

We will show $\Phi_2 \circ \Phi_1 = \phi$. First coordinate:
\begin{gather*}
\frac{2\left(\frac{x_2(1-dx_1^2)}{x_1^2}+\frac{1}{x_1^2}\right)-a-d}{a-d} = \frac{ax_1^2+dx_1^2-2-2x_2+2dx_1^2x_2}{(d-a)x_1^2}\\
\text{Substitute $ax_1^2-1 = dx_1^2x_2^2-x_2^2$:}\\
\frac{dx_1^2x_2^2-x_2^2-1-2x_2+dx_1^2+2dx_1^2x_2}{dx_1^2-ax_1^2} = \frac{(x_2+1)^2(dx_1^2-1)}{dx_1^2-ax_1^2}\\
\text{Substitute $ax_1^2 = dx_1^2x_2^2-x_2^2+1$:}\\
\frac{(x_2+1)^2(dx_1^2-1)}{dx_1^2-dx_1x_2^2+x_2^2-1} = \frac{(x_2+1)^2(dx_1^2-1)}{(dx_1^2-1)(-x_2^2+1)} = \frac{(x_2+1)^2}{(1-x_2)(1+x_2)} = \frac{1+x_2}{1-x_2}
\end{gather*}
Second coordinate is clearly $\frac{1+x_2}{x_1(1-x_2)}$ since it is the first coordinate multiplied by $\frac{1}{x_1}$.

This proves $\Phi_2 \circ \Phi_1 = \phi$. Therefore $\phi$ is a $K$-rational map $C_2 \rightarrow C_1$.

Now we want to prove the birational equivalence e.g. $\phi \circ \psi = id_{C_1}$ and $\psi \circ \phi = id_{C_2}$. Using the the fact that these maps are compositions of maps mentioned and from $W.2$ we know that $\Phi_1 \circ \Psi_2 = id_{C'}$ and from $G.5$ we know $\Phi_2 \circ \Psi_1 = id_{C_1}$:
\begin{gather*}
\phi \circ \psi = (\Phi_2 \circ \Phi_1) \circ (\Psi_2 \circ \Psi_1) = \Phi_2 \circ (\Phi_1 \circ \Psi_2) \circ \Psi_1 = \Phi_2 \circ id_{C'} \circ \Psi_1 = id_{C_1}
\end{gather*}
Similarly from $G.5$ we know that $\Psi_1 \circ \Phi_2 = id_{C'}$ and from $W.2$ we know $\Psi_2 \circ \Phi_1 = id_{C_2}$
\begin{gather*}
\psi \circ \phi =  (\Psi_2 \circ \Psi_1) \circ (\Phi_2 \circ \Phi_1) = \Psi_2 \circ (\Psi_1 \circ \Phi_2) \circ \Phi_1 = \Psi_2 \circ id_{C'} \circ \Phi_1 = id_{C_2}
\end{gather*}

$\qed$

\textbf{Problem:}

Assume $F/K$ EFF can be given by $y^2=(x^2-a)(x^2-d)$ since $F$ is EFF $f(x)=(x^2-a)(x^2-d)$ must be separable and therefore we can assume $a,d \in K^*, a \neq d$. Assume there exists $c \in K^*: c^2=a$. Then $F/K$ is given by $y^2=(x-c)(x+c)(x^2-d)$.

We will investigate the structure of divisor $(x-c)$. By Q.4 we know that $[F:K(x)]=2$, since $c \in K: K(x)=K(x-c)$ and we can see that $\deg((x\pm c)_{-})=2=\deg((x\pm c)_{+})$. By Q.5 we know there exist distinct places $P,Q$ of degree one s.t. $(x)_{-}=P+Q, (y)_{-}=2P+2Q$. We then have:
\begin{gather*}
2v_P(y)=v_P(x-c)+v_P(x+c)+v_P(x^2-d)\\
v_P(x) < 0 \implies v_P(x^2-d)=v_P(x^2)=2v_P(x)=v_P(y) \implies\\
-2 = v_P(y) = v_P(x-c)+v_P(x+c)\\
\text{And same for $Q$:}\\
-2 = v_Q(y) = v_Q(x-c)+v_Q(x+c)\\
\end{gather*}
Since $\deg((x\pm c)_{-})=2$ the only possibilities are:
\begin{gather*}
v_P(x-c) = -1 = v_P(x+c) \text{ and } v_Q(x-c) = -1 = v_Q(x+c) \\
\text{ or w.l.o.g.:}\\
v_P(x-c) = -2, v_P(x+c) = 0, v_Q(x-c)=0, v_Q(x+c) = -2
\end{gather*}
The first option implies $(x-c)_{-}=(x+c)_{-}=P+Q$ and the second one $(x-c)_{-}=2P, (x+c)_{-}=2Q$. The second option is not possible since $\{1,x\}$ forms a basis of $\mathcal{L}(P+Q)$ and therefore $x \pm c \in \mathcal{L}(P+Q)$ but $(x-c)_{-}=2P \implies x-c \notin \mathcal{L}(P+Q)$ which is a contradiction.

Now we know $(x-c)_{-}=(x+c)_{-}=P+Q$. There exists a place $W$ different from $P,Q$ s.t. $v_W(x-c) > 0$. Again we get an equality:
\begin{gather*}
2v_W(y) = v_W(x-c)+v_W(x+c)+v_W(x^2-d)
\end{gather*}
Since we know $(y)_{-}=2P+2Q$ it must be $v_W(y) \geq 0$. For similar reasons it must be $v_W(x+c) \geq 0$. Also $v_W(x) \geq 0$ since $(x)_{-} = P+Q$ using valuation definition we have $v_W(x^2-d) \geq min(2v_W(x), v_W(d))$. $d \in K \implies v_W(d)=0$ and with $v_W(x) \geq 0$ we get $v_W(x^2-d) \geq 0$. This gives us $v_W(y) \geq v_W(x-c) \implies v_W(y) \geq 1$. 

Either $v_W(x-c) = 2$ then $(x-c)_{+}=2W$ since $\deg((x-c)_{+})=2$ and $v_W(y) \geq 1$ or $v_W(x-c)=1$ which means there exists another place $T$ s.t. $v_T(x-c)=1 \implies (x-c)_{+}=W+T$ and $v_T(y) \geq 1$.

In both cases clearly $v_W(x+c) = 0$ since it must be nonnegative integer and $v_W(x+c) > 0$ implies $c \in W$ and $c \in K$ which is a contradiction. Therefore we can say:
\begin{gather*}
2v_W(y) = v_W(x-c)+v_W(x^2-d)
\end{gather*}

If $(x-c)_{+} = 2W$ then $(x-c)=2W-P-Q \implies (x-c)^2 = 4W - 2P - 2Q$. Set $b = \frac{c^2-d}{2}$ then by Q.8 we have $(y+x^2+b)=2P-2Q \implies (x-c)^2(y+x^2+b) = 4W - 4Q = 4[W-Q]$. Now we need to show that $2W-2Q \neq (t)$ for some $t \in F$. Assume there exists $t \in F$ s.t. $(t) = 2W-2Q$. By Q.5 $(y+x^2)_{-}=2Q$ then $y+x^2, t \in \mathcal{L}(2Q)$. $\{1,t\}$ forms a basis of $\mathcal{L}(2Q)$ which means there exist $s_1,s_2 \in K$ s.t. $y+x^2 = s_1t +s_2$...

Converse implication. We have $F/K$ EFF s.t. $\text{Pic}^0(F/K)$ contains an element of order 4. $F/K$ is EFF therefore there exists a place $Q$ of degree 1. Set $[D] \in \text{Pic}^0(F/K)$ to be the element of order 4 i.e. $4[D] \in \text{Princ}(F/K)$. Also there exists a place $W$ of degree 1 s.t. $[D]=[W-Q] \implies 4W-4Q = (t), t \in F$. $2[D]=[2D]$ in involution. Denote $P$ the place of degree 1 s.t. $[2D] = [P-Q]$. 

Using Q.7 and Q.8 we get that $F/K$ is given by $y^2=g(x^2), g(x)=x^2+2bx+c$ where $b,c \in K$. Also that $(x)_{-}=P+Q, (y)_{-}=2P+2Q$.
\end{document}