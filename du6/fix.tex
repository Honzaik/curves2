\documentclass[12pt, a4paper]{article}
\usepackage[margin=1in]{geometry}
\usepackage[utf8x]{inputenc}
\usepackage{indentfirst} %indentace prvního odstavce
\usepackage{mathtools}
\usepackage{amsfonts}
\usepackage{amsmath}
\usepackage{amssymb}
\usepackage{graphicx}
\usepackage{enumitem}
\usepackage{subfig}
\usepackage{float}
\usepackage[czech]{babel}
\usepackage{mathdots}
\usepackage{slashbox}

\begin{document}
\begin{gather*}
\frac{y^2-x^2+x+1}{-y(x^2-x-2)} = \frac{(x+2)^3-2(x+2)^2+(x+2)}{-y(x^2-x-2)} = \frac{-y((x+2)^3-2(x+2)^2+(x+2))}{((x+2)^3-(x+2)^2+(x+2))(x^2-x-2)}\\
\frac{-y((x+2)^2-2(x+2)+1)}{((x+2)^2-(x+2)+1)(x^2-x-2)}
\end{gather*}
\begin{gather*}
\text{vynásobíme všechny  $A_i$: $Z$}\\
A_1 \rightarrow YZ^2(Y^2-X^2+XZ+Z^2) = YZ((X+2Z)^3 - 2(X+2Z)^2Z+(X+2Z)Z^2) \rightarrow\\
YZ((X+2Z)^2 - 2(X+2Z)Z+Z^2)\\
A_2 = -Y^2Z(X^2-XZ-2Z^2) = -((X+2Z)^3 - (X+2Z)^2Z+(X+2Z)Z^2)(X^2-XZ-2Z^2) \rightarrow\\
-((X+2Z)^2 - (X+2Z)Z+Z^2)(X^2-XZ-2Z^2)\\
A_3 = Z^2Y(X+2Z)^2 \rightarrow Z^2Y(X+2Z)
\end{gather*}
\end{document}

