\documentclass[12pt, a4paper]{article}
\usepackage[margin=1in]{geometry}
\usepackage[utf8x]{inputenc}
\usepackage{indentfirst} %indentace prvního odstavce
\usepackage{mathtools}
\usepackage{amsfonts}
\usepackage{amsmath}
\usepackage{amssymb}
\usepackage{graphicx}
\usepackage{enumitem}
\usepackage{subfig}
\usepackage{float}
\usepackage[czech]{babel}
\usepackage{mathdots}
\usepackage{slashbox}

\begin{document}
\begin{center}
\large NMMB538 - DÚ5

\normalsize Jan Oupický
\end{center}
\vspace{1\baselineskip}

\section{}
\begin{enumerate}
    \item Použijeme lemma R.1. Využijeme předchozího úkolu, kde $F' = K(D)$. Pokud označíme $y_1 \coloneqq \frac{x_2^2 + (g)}{x_1^2 + (g)} = t^2 = u \in K(D), y_2 \coloneqq \frac{x_2(b-x_1^2) + (g)}{x_1^2 + (g)} = st = v \in K(D)$. $\phi$ dle R.1 existuje pokud $y_1$ nebo $y_2$ je transcendentní nad $K$ a $f(y_1,y_2)=0$. Z předchozího úkolu zřejmě platí, že oba prvky jsou transcendentní nad $K$. Dále jsme v minulém úkolu také ukázali, že platí $f(y_1,y_2)=f(u,v) \in F' = K(D)$.

    \item Dle R.7 platí $\phi = \sigma^*$, kde $\sigma=(\sigma_1, \sigma_2)$, kde $\sigma_1 = y_1, \sigma_2 = y_2$. Zvolíme reprezentanty $(y_1,y_2)$ například jako $r_1 = \frac{x_2^2}{x_1^2}, r_2 = \frac{x_2(b-x_1^2)}{x_1^2}$. 

    \item Víme, že $y_1,y_2$ jsou trans. nad $K$, tedy $\deg(\sigma) < \infty$. Zároveň jsme ukázali, že $K(y_1,y_2)=K(u,v)$ a v minulých úkolech jsme ukázali, že $[F':K(y_1,y_2)] = 2 \implies \text{ protože }F'=K(D)\implies [K(D):K(y_1,y_2)]=2=\deg(\sigma)$.

    \item Pro přehlednost označme $P_\infty$ ze zadání jako $M_\infty \in \mathbb{P}_{K(C)/K}$. Místo $M_\infty \in \mathbb{P}_{K(C)/K}$ zřejmě obsahuje $x_1^{-2}+(f)$, jelikož $v_{\infty}(x_1+(f))=-2$. Tento prvek se nám pomocí $\sigma^*$ zobrazí na $\frac{x_1^4+(g)}{x_2^4+(g)} = \frac{x_2^{-4}+(g)}{x_1^{-4}+(g)} = u^{-2} \in \text{Im}(\sigma^*) = F \subset F'$. Z minulého úkolu víme, že jediné místo $F/K$, co obsahuje $u^{-2}$ je $P_\infty \in \mathbb{P}_{F/K}$, tedy $\sigma^*(M_\infty)=P_\infty$. Dále jsme také ukázali, že pro $P'_0 \in \mathbb{P}_{K(D)/K}$ platí $P'_0 | P_\infty \implies P'_0 | \sigma^*(M_\infty)$.

    \item $\Rightarrow$:

    Opět použijeme značení $M_\alpha$ pro místo $K(C)/K$. Zvolme $\rho \in K(C)$. Víme, že platí $v_{M_\alpha}(\rho) > 0 \iff \rho(\alpha) = 0$. Z předpokladů platí obdobně, že $v_{P'_\beta}(\sigma^*(\rho)) > 0 \implies \sigma^*(\rho)(\beta) = 0$. Dle lemma R.8 pokud je $\rho(\sigma(\beta))$ definované, tak platí $\rho(\sigma(\beta))=\sigma^*(\rho)(\beta) \implies\rho(\sigma(\beta)) = 0 \implies \rho \in M_{\sigma(\beta)}$, $\sigma(\beta) \in C$ tedy musí platit $M_{\sigma(\beta)} = M_\alpha \iff \sigma(\beta) = \alpha$. 

    Pokud $\rho(\sigma(\beta))$ není def. tak musí platit $\beta \notin \text{Dom}(\sigma) \implies \beta_1 = 0 \implies \beta_2 = 0$. Případ $\sigma(\beta) \notin \text{Dom}(\rho)$ nenastane, jelikož $\rho$ můžeme brat jako $\frac{x_1-\alpha_1+(f)}{1+(f)}$ nebo $\frac{x_2-\alpha_2+(f)}{1+(f)}$ tedy $\text{Dom}(\rho)=C$. Tedy v případě $\beta=(0,0)$ máme z 4) $P'_0 | P_\infty$. 

    Dále by tedy $P'_0$ obsahovalo místo $P_\alpha$ což je spor, protože pokud $\alpha_1 \neq 0$, tak $v_{P_\alpha}(u-\alpha_1+(w_D))>0$ a zároveň $v_{P'_0}(u-\alpha_1+(w_D))=\min\{v_{P'_0}(u+(w_D)), v_{P'_0}(-\alpha_1+(w_D))\}=-2$. 

    Pokud $\alpha_1 = 0$ tak $\alpha_2 = 0$ a v minulém úkolu, jsme ukázali, že $P'_0$ neobsahuje $P_0$.

    $\Leftarrow$:

    Obdobně zvolme $\rho \in M_\alpha \implies \rho(\alpha) = 0$. Z předpokladu víme, že $\beta \in $ Dom$(\sigma)$, tedy je $\rho(\sigma(\beta))$ definováno, jelikož Dom$(\rho) = C$. Platí tedy $0 = \rho(\alpha) = \rho(\sigma(\beta)) = \sigma^*(\rho)(\beta)$. Neboli $v_{P'_\beta}(\sigma^*(\rho)) > 0 \implies P'_\beta | \sigma^*(M_\alpha)$.

    \item Vidíme, že $D \setminus \{(0,0)\} = \text{Dom}(r_1) \cap \text{Dom}(r_2)$. Kdyby $\text{Dom}(\sigma) \supset \{(0,0)\}$, tak by dle 5) $P'_0|P_{(0,0)}$ což nejde. Takže $\text{Dom}(r_1) \cap \text{Dom}(r_2)=$ Dom$(\sigma)$. 
\end{enumerate}

\section{}
Pokud $\alpha$ je kořen polynomu $x^3+a_4x+a_6$, tak substitucí $x \mapsto x+\alpha$ dostaneme $x^3+3\alpha x^2+(3\alpha^2+a_4)x+\alpha^3+a_4\alpha+b$, z definice $\alpha$ platí $\alpha^3+a_4\alpha+b = 0$, máme tedy polynom $x^3+3\alpha x^2+(3\alpha^2+a_4)x$.

\section{}
Označme $w_{\tilde{D}}(x,y)=y^2-x^3+x-1$. Kořen $x^3-x+1$ v $\mathbb{Z}_5$ je 3. Máme tedy $w_D(x,y)=w_{\tilde{D}}(x+3,y)$ neboli $w_D(x,y)=y^2-x^3-4x^2-x$.

Máme tedy $a=4, b=1 \implies w_C(x,y)=y^2-x^3-2x^2-2x$ a následně \\$w_{\tilde{C}}(x,y)=w_C(x-\frac{-3}{3},y)=w_C(x+1,y)=y^2-x^3-4x$. Dohromady:
\begin{gather*}
w_{\tilde{D}}(x,y)=y^2-x^3+x-1\\
w_D(x,y)=y^2-x^3-4x^2-x\\
w_C(x,y)=y^2-x^3-2x^2-2x\\
w_{\tilde{C}}(x,y)=y^2-x^3-4x
\end{gather*}

Z cvičení 1 známe $K$-rational map $\sigma': D \rightarrow C$, $\deg(\sigma')=2$. Sestrojíme $\sigma$ jako složení $\sigma_2 \circ \sigma' \circ \sigma_1: \tilde{D} \rightarrow D \rightarrow C \rightarrow \tilde{C}$.

Dle R.9 pokud $\sigma_1, \sigma_2$ budou konečného stupně, tak bude i $\sigma$. Z prvního cvičení víme, že $\sigma'=\left(\frac{x_2^2+(w_D)}{x_1^2+(w_D)}, \frac{x_2(1-x_1^2)+(w_D)}{x_1^2+(w_D)}\right)$.

Nyní popíšeme $\sigma_1$. Sestrojme $\sigma_1^*: K(D) \rightarrow K(\tilde{D})$. Z výše udělané substituce zřejmě: 
\[\sigma_1^*(x+(w_D)) = x+2+(w_{\tilde{D}})\]
\[\sigma_1^*(y+(w_D)) = y+(w_{\tilde{D}})\]

A následně tedy $\sigma_1 = (x+2+(w_{\tilde{D}}),y+(w_{\tilde{D}}))$. 

Obdobně pro $\sigma_2$: $\sigma_2^*: K(\tilde{C}) \rightarrow K(C), w_C(x,y)=w_{\tilde{C}}(x+4,y) \implies$
\[\sigma_2^*(x+(w_{\tilde{C}}))=x+4+(w_C)\]
\[\sigma_2^*(y+(w_{\tilde{C}}))=y+(w_C)\]

$\sigma_2=(x+4+(w_C), y+(w_C))$.

Zřejmě jsou $x+2+(w_{\tilde{D}}), y+(w_{\tilde{D}}), x+4+(w_C), y+(w_C)$ transcendentní nad $K$, tedy $\sigma_1,\sigma_2$ jsou konečného stupně. To, že $\sigma'$ je konečného stupně jsme ukázali v 1). Následně tedy $\sigma$ je také konečného stupně dle R.9.

Zřejmě Dom$(\sigma_1) = \tilde{D}$, Dom$(\sigma_2)=C$. 

Z prvního cvičení víme, že Dom$(\sigma')=D\setminus \{(0,0)\}$.

Neboli pro $\alpha=(\alpha_1,\alpha_2) \in \text{Dom}(\sigma)$:\\
\[\sigma(\alpha)=\sigma_2(\sigma'(\sigma_1(\alpha)))=\left(\frac{\alpha_2^2}{(\alpha_1+2)^2}+4,\frac{\alpha_2(1-(\alpha_1+2)^2)}{(\alpha_1+2)^2}\right)\]

\section{}
$\mathbb{Z}_5$-racionální body $\tilde{D}$ jsou:
\[(0,1),(0,4),(1,1),(1,4),(3,0),(4,1),(4,4)\]

$\mathbb{Z}_5$-racionální body $\tilde{C}$ jsou:
\[(0,0),(1,0),(2,1),(2,4),(3,2),(3,3),(4,0)\]

Jelikož ve cvičení 3 jsme určili, že Dom$(\sigma_1) = \tilde{D}$, Dom$(\sigma_2) = \tilde{C}$ a \\
Dom$(\sigma') = D \setminus \{(0,0)\}$. Prvek $\tilde{D}$, který nebude prvkem Dom$(\sigma)$ je ten, co se pomocí $\sigma_1$ zobrazí na $(0,0)$. Z definice to je pouze bod $(3,0)$, jelikož $\sigma_1((3,0)) = (3+2,0) = (0,0)$. Jiný takový bod není.

Zřejmě $\tilde{D} \supset \text{Dom}(\sigma) \implies \tilde{D} \cap \text{Dom}(\sigma) = \text{Dom}(\sigma)$ a Dom$(\sigma) = \tilde{D}\setminus \{(3,0)\}$. 

Spočteme tedy obrazy prvků $\{(0,1),(0,4),(1,1),(1,4),(4,1),(4,4)\}$:
\begin{gather*}
\sigma((0,1)) = \left(\frac{1^2}{(0+2)^2}+4,\frac{1(1-(0+2)^2)}{(0+2)^2}\right) = (3, 3)\\
\sigma((0,4)) = \left(\frac{4^2}{(0+2)^2}+4,\frac{4(1-(0+2)^2)}{(0+2)^2}\right) = (3, 2)\\
\sigma((1,1)) = \left(\frac{1^2}{(1+2)^2}+4,\frac{1(1-(1+2)^2)}{(1+2)^2}\right) = (3, 3)\\
\sigma((1,4)) = \left(\frac{4^2}{(1+2)^2}+4,\frac{4(1-(1+2)^2)}{(1+2)^2}\right) = (3, 2)\\
\sigma((4,1)) = \left(\frac{1^2}{(4+2)^2}+4,\frac{1(1-(4+2)^2)}{(4+2)^2}\right) = (0, 0)\\
\sigma((4,4)) = \left(\frac{4^2}{(4+2)^2}+4,\frac{4(1-(4+2)^2)}{(4+2)^2}\right) = (0, 0)\\
\end{gather*}
\end{document}

