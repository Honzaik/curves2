\documentclass[12pt, a4paper]{article}
\usepackage[margin=1in]{geometry}
\usepackage[utf8x]{inputenc}
\usepackage{indentfirst} %indentace prvního odstavce
\usepackage{mathtools}
\usepackage{amsfonts}
\usepackage{amsmath}
\usepackage{amssymb}
\usepackage{graphicx}
\usepackage{enumitem}
\usepackage{subfig}
\usepackage{float}
\usepackage[czech]{babel}
\usepackage{mathdots}
\usepackage{slashbox}

\begin{document}
\begin{center}
\large NMMB538 - DÚ2

\normalsize Jan Oupický
\end{center}
\vspace{1\baselineskip}

\section{}
Chceme ukázat, že $F/F^{p^i}, i \geq 0, p = char(F)$ je čistě neseparabilní. Toto tělesové rozšíření je zřejmě algebraické, jelikož $\forall \alpha \in F: m_{\alpha,F^{p^i}} = x^{p^i} - \alpha^{p^i} \in F^{p^i}[x]$. Chceme tedy ukázat, že $\forall \alpha \in F$ je čistě neseparabilní. 

Využijeme prop. S.5 implikaci $(ii) \implies (i)$. Čistě z definice $F^{p^i}$ tedy dokážeme nalézt dané $j \geq 0$ tž. $\alpha^{p^{i^{j}}} \in F^{p^i}$ $(j = 1)$. Rozšíření je tedy čistě neseparabilní.

\section{}
Máme $char(K) = p$. Předpokládejme, že $K$ je perfektní, neboli $a \mapsto a^p$ je automorfimus $K$. Máme tedy $F = K(x), F^p = (K(x))^p$. Víme, že platí $a,b \in K[x]: (a+b)^p = a^p + b^p$. Tudíž $f(x) = \sum f_ix^i \in K[x] \implies (f(x))^p = \sum f_i^p x^{ip}$. Díky tomu, že je $K$ perfektní, víme $\forall a \in K \exists b \in K: b^p = a$. Poté již nahledéneme, že $(K(x))^p = K(x^p) = F^p$.

Chceme tedy spočítat $[F:F^p]=[K(x):K(x^p)]$. $x$ je algebraický prvek nad $K(x^p)$, protože $g(T) = T^p - x^p \in K(x^p)[T] = F^p[T]$. Tento polynom je $m_{x,F^p}$, protože kdyby existoval $f \in F^p[T]: deg(f) < deg(g)$, tak by $f|g$. Zároveň ale $g(T)=T^p - x^p = (T-x)^p$, takže by $f$ musel být polynom, který je tvaru $(T-x)^i, i < p$, ale to nemůže být polynom $F^p[T]=K(x^p)[T]$, protože $x^i$ se v tam nevyskytují.

Zároveň zřejmě $K(x^p)(x)=K(x,x^p)=K(x)$, takže 
\[
p = \deg m_{x,F^p} = [K(x):K(x^p)] = [F:F^p]
\]

Není důvod proč stejný postup nebude fungovat pro $[F:F^{p^i}]$, takže $[F:F^{p^i}] = p^i$. 

Díky perfektnosti $K$, $(K(x,y))^p = K(x^p,y^p)$. Počítáme tedy $[K(x,y):K(x^p,y^p)] = [K(x,y):K(x,y^p)] \cdot [K(x,y^p):K(x^p,y^p)]$. Definujme $D= K(x) \implies [K(x,y):K(x,y^p)] = [D(y):D(y^p)] = p$. Toto platí díky předchozí části ($D = K, y=x$). Stejně tak můžeme napsat $D = K(y^p) \implies  [K(x,y^p):K(x^p,y^p)] = [D(x):D(x^p)]=p$. Takže $[K(x,y):K(x^p,y^p)] = p^2$.

Nyní spočteme hodnoty $N_{F|F_p}(\alpha), \alpha=x^2+1$ a $Tr_{F|F_p}(\alpha)$. Víme, že $x$ je čistě neseparabilní. Tudíž $[F:F^p]_s < [F:F^p] (\iff [K(x^p)(x):K(x^p)]_s < [K(x^p)(x):K(x^p)])$. Dále máme rovnost $[F:F^p] = [F:F^p]_s \cdot [F:F^p]_i = p \implies [F:F^p]_s = 1, [F:F^p]_i = p$. Pro výpočet normy a stopy použijeme tedy prop S.12, kde $s=1, t=p$. Jediný prvek $\text{Hom}_{F^p}(F,\bar{F^p})$ je tedy identita na $F$. Takže $\sigma(\alpha)=\alpha \implies N_{F|F_p}(\alpha) = \alpha^p = (x^2+1)^p = x^{2p}+1, Tr_{F|F_p}(\alpha) = p(x^2+1) = 0$. 

Nyní předpokládejmě, že $K$ není perfektní. Tudíž musí být $K$ nekonečné těleso s charakteristikou $p$, kde Frobeinův endomorfismus není surjektivní. Tudíž $K(x)^p \neq K(x^p)$. Poté rozšíření $F/F^p$ nebude konečného stupně, jelikož v $K(x)$ existuje nekonečně mnoho prvků z $K$, které nejsou tvaru $a^p, a \in K$ tudíž nejsou v $F^p$.

\section{}
Mějme tedy $K \subset L$ separabilní rozšíření těles. Dokážeme $L$ perfektní $\iff K$ perfektní. Platí $char(K) = 0 \iff char(L) = 0$, tedy v případě nulové charakteristiky je to zřejmé. Uvažujme tedy $p = char(K) = char(L)$.

$\Rightarrow$: $L$ je perfektní, tudíž je Frobeinův endomorfimus surjektivní na $L$. Zároveň $\forall a \in K: a^p \in K$, tudíž Frobeinův endomorfimus nemůže zobrazit prvek $a \in K \subset L$ na prvek, který je mimo $K$. Takže je Frobeinův endomorfismus surjektivní i na $K$ neboli $K$ je perfektní.

$\Leftarrow$: To, že je Frobeinův endomorfimus surjektivní můžeme vyjádřit, že $K = K^p$. Tedy $K$ je perfektní $\iff$ $K^p = K$. Dále z definice separability platí, že $L/K$ je algebraické rozšíření. Algebraické rozšíření můžeme zapsat takto $L = \cup_{a\in L} K(a)$. $a$ je algebraické nad $K$ a tedy $K(a)$ je rozšíření konečného stupně. Pokud ukážeme, že $K(a)$ je perfektní, tak bude i $L$ perfektní, jelikož je to sjednocení perfektních těles.

Označme $[K(a):K]=n \in \mathbb{N}$. Označme $f(x)=m_{a,K}$. Díky perfektnosti $K$ platí $(K(a))^p = K(a^p), K = K^p$, $f(a)=0 \implies (f(a))^p = \iff f(a^p)=0$ tedy $[K(a^p):K] = n$. Máme tedy:
\[
n=[K(a):K]=[K(a):K(a^p)]\cdot [K(a^p):K], [K(a^p):K] =n \implies [K(a):K(a^p)] = 1
\]

Neboli $K(a)=K(a^p)=(K(a))^p$ tedy $K(a)$ je perfektní. Takže $L$ je perfektní.

\section{}
Označme $w(x,y) = y^2 + yg(x) - f(x)$ Weierstrasuv polynom. Z definice a minulých přednášek víme, že $[F:K(x)]=2$ a $[F:K(y)] = 3$, jelikož $m_{y,K(x)}(T) = w(x,T) = T^2 + Tg(x) - f(x) \in K(x)[T]$. Víme, že tento polynom je ireducibilní a zároveň platí $w(x,y) = 0$ v $F$. Stejně tak $m_{x,K(y)}(T) = w(T,y) = -T^3 + a_2T^2 + T(a_4+a_1) + a_6 + a_3y + y^2 \in K(y)[T]$. 

$F/K(x)$ je čistě neseparabilní pokud všechny prvky $F$ jsou čistě neseparabilní nad $K(x)$. Všechny prvky  z $F \cap K(x)$ jsou čistě neseparabilní nad $K(x)$ z definice. 

Chceme tedy určit, kdy je $y$ čistě neseparabilní nad $K(x)$. Dle Prop. S.5 musí být $min_{y,K(x)}(T)$ tvaru $T^{p^j}-\beta, \beta \in K(x)$. Výše vidíme, že toto může nastat pouze v případě, kdy $p = char(K) = 2$ a $g(x)=0 \iff a_1=a_3=0$. V jiných případech není $y$ čistě separabilní nad $K(x)$, tedy ani $F$.

Ukážeme, že v tomto případě je křivka určená tímto polynomem $w(x,y)= y^2 - f(x)$ má singularitu:
\[ 
\frac{\partial w}{\partial x}(x,y) = -3x^2+2a_2x+a_4 = x^2+a_4
\]
\[
\frac{\partial w}{\partial y}(x,y) = 2y = 0
\]
Chceme bod $(x_1,y_1)$ ve kterém jsou derivace výše $0$ a splňuje rovnici $w(x_1,y_1)=0$. $x_1$ volíme dle hodnoty $a_4$, pokud $a_4 = 0 \implies x_1 = 0$ a naopak. Nyní obě derivace jsou 0. Chceme ještě aby platilo $y_1^2-f(x_1) = 0 \implies y_1^2 = f(x_1)$. Zřejmě lze zvolit $y_1$ aby toto platilo, tudíž máme singularitu.

Nyní uvažme stejný postup pro určení, kdy je $x$ čistě neseparabilní nad $K(y)$. Musí tedy platit $p=3=char(K)$ a $a_2=0, a_4+a_1 = 0$. Poté je $m_{x,K(y)}(T)=-T^3+a_6 + a_3y + y^2 \in K(y)[T]$.
Opět určíme singularitu:
\[ 
\frac{\partial w}{\partial x}(x,y) = -3x^2 = 0
\]
\[
\frac{\partial w}{\partial y}(x,y) = a_3+2y = a_3 - y
\]

Pro bod singularity $(x_1,y_1)$ tedy platí $y_1 = a_3$ a $x_1$ volíme, aby platilo $x_1^3=a_6+2a_3$.

V případě, kdy $y$ není čistě neseparabilní nad $K(x)$, tak $y$ je separabilní nad $K(x)$ dle Prop S.4. Zároveň prvky $F \cap K(x)$ jsou tedy také separabilní dle S.4 a ostatní jsou jejich kombinace, které jsou také separabilní, jelikož separabilní prvky tvoří těleso. Takže je $F/K(x)$ separabilní rozšíření. Stejný argument se dá použít proč je $F/K(y)$ separabilní v případě, když $x$ není čistě neseparabilní.

\section{}
Výše jsme určili $[F:K(x)]=2$ a $[F:K(y)] = 3$, zvolme bázi $B_y = (1,y)$ tělesa $F$ nad $K(x)$ (pro zjednodušení místo $\frac{y+(w)}{1+(w)} \in F$ prvek $y$, stejně tak pro ostatní zmíněné prvky). Bázi $F$ nad $K(y)$ zvolíme $B_x=(1,x,x^2)$ (obdobné ztotožnění).

Spočteme tedy matici $M_x$:
\begin{gather*}
x\cdot 1 = x \implies \mu_1 = (0, 1, 0)\\
x\cdot x = x^2 \implies \mu_2 = (0, 0, 1)\\
x\cdot x^2 = x^3 = a_2x^2+x(a_4+a_1) + a_6 + a_3y + y^2 \implies \mu_3 = (a_6 + a_3y + y^2, a_4+a_1, a_2)\\
M_x = \begin{pmatrix}
0 & 0 & a_6 + a_3y + y^2\\
1 & 0 & a_4+a_1\\
0 & 1 & a_2
\end{pmatrix}
\end{gather*}
Takže $N_{F|K(y)}(x) = a_6 + a_3y + y^2, Tr_{F|K(y)}(x)=a_2$. Obdobně pro $M_y$:

\begin{gather*}
y\cdot 1 = y \implies \mu_1 = (0, 1)\\
y\cdot y = y^2 = f(x)-yg(x) \implies \mu_2 = (f(x),-g(x))\\
M_y = \begin{pmatrix}
0 & f(x)\\
1 & -g(x)
\end{pmatrix}
\end{gather*}
Takže $N_{F|K(x)}(y) = -f(x), Tr_{F|K(x)}(y)=-g(x)$.

\end{document}

