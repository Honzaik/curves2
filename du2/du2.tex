\documentclass[12pt, a4paper]{article}
\usepackage[margin=1in]{geometry}
\usepackage[utf8x]{inputenc}
\usepackage{indentfirst} %indentace prvního odstavce
\usepackage{mathtools}
\usepackage{amsfonts}
\usepackage{amsmath}
\usepackage{amssymb}
\usepackage{graphicx}
\usepackage{enumitem}
\usepackage{subfig}
\usepackage{float}
\usepackage[czech]{babel}
\usepackage{mathdots}
\usepackage{slashbox}

\begin{document}
\begin{center}
\large NMMB538 - DÚ2

\normalsize Jan Oupický
\end{center}
\vspace{1\baselineskip}

\section{}
Chceme ukázat, že $F/F^{p^i}, i \geq 0, p = char(F)$ je čistě neseparabilní. Toto tělesové rozšíření je zřejmě algebraické, jelikož $\forall \alpha \in F: m_{\alpha,F^{p^i}} = x^{p^i} - \alpha^{p^i} \in F^{p^i}[x]$. Chceme tedy ukázat, že $\forall \alpha \in F$ je čistě neseparabilní. 

Využijeme prop. S.5 implikaci $(ii) \implies (i)$. Čistě z definice $F^{p^i}$ tedy dokážeme nalézt dané $j \geq 0$ tž. $\alpha^{p^{i^{j}}} \in F^{p^i}$ $(j = 1)$. Rozšíření je tedy čistě neseparabilní.

\section{}
Máme $char(K) = p$. Předpokládejme, že $K$ je perfektní, neboli $a \mapsto a^p$ je automorfimus $K$. Máme tedy $F = K(x), F^p = (K(x))^p$. Víme, že platí $a,b \in K[x]: (a+b)^p = a^p + b^p$. Tudíž $f(x) = \sum f_ix^i \in K[x] \implies (f(x))^p = \sum f_i^p x^{ip}$. Díky tomu, že je $K$ perfektní, víme $\forall a \in K \exists b \in K: b^p = a$. Poté již nahledéneme, že $(K(x))^p = K(x^p) = F^p$.

Chceme tedy spočítat $[F:F^p]=[K(x):K(x^p)]$. $x$ je algebraický prvek nad $K(x^p)$, protože $g(T) = T^p - x^p \in K(x^p)[T] = F^p[T]$. Tento polynom je $m_{x,F^p}$, protože kdyby existoval $f \in F^p[T]: deg(f) < deg(g)$, tak by $f|g$. Zároveň ale $g(T)=T^p - x^p = (T-x)^p$, takže by $f$ musel být polynom, který je tvaru $(T-x)^i, i < p$, ale to nemůže být polynom $F^p[T]=K(x^p)[T]$, protože $x^i$ se v tam nevyskytují.

Zároveň zřejmě $K(x^p)(x)=K(x,x^p)=K(x)$, takže 
\[
p = \deg m_{x,F^p} = [K(x):K(x^p)] = [F:F^p]
\]

Není důvod proč stejný postup nebude fungovat pro $[F:F^{p^i}]$, takže $[F:F^{p^i}] = p^i$. Stejně tak symetricky můžeme nově definovat $D=K(x)$ a hodnotu $[K(x,y):K(x^p,y^p,x)]=[D(y):D(y^p)]$ spočítat obdobně. Máme tedy $[K(x,y):K(x^p,y^p)] = p^2$.

Nyní spočteme hodnoty $N_{F|F_p}(\alpha), \alpha=x^2+1$ a $Tr_{F|F_p}(\alpha)$. Víme, že $x$ je čistě neseparabilní. Tudíž $[F:F^p]_s < [F:F^p] (\iff [K(x^p)(x):K(x^p)]_s < [K(x^p)(x):K(x^p)])$. Dále máme rovnost $[F:F^p] = [F:F^p]_s \cdot [F:F^p]_i = p \implies [F:F^p]_s = 1, [F:F^p]_i = p$. Pro výpočet normy a stopy použijeme tedy prop S.12, kde $s=1, t=p$. Jediný prvek $\text{Hom}_{F^p}(F,\bar{F^p})$ je tedy identita na $F$. Takže $\sigma(\alpha)=\alpha \implies N_{F|F_p}(\alpha) = \alpha^p = (x^2+1)^p = x^{2p}+1, Tr_{F|F_p}(\alpha) = p(x^2+1) = 0$. 

Nyní předpokládejmě, že $K$ není perfektní. Tudíž musí být $K$ nekonečné těleso s charakteristikou $p$, kde Frobeinův endomorfismus není surjektivní. Tudíž $K(x)^p \neq K(x^p)$. Poté rozšíření $F/F^p$ nebude konečného stupně, jelikož v $K(x)$ existuje nekonečně mnoho prvků z $K$, které nejsou tvaru $a^p, a \in K$ tudíž nejsou v $F^p$.

\end{document}

