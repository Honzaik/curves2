\documentclass[12pt, a4paper]{article}
\usepackage[margin=1in]{geometry}
\usepackage[utf8x]{inputenc}
\usepackage{indentfirst} %indentace prvního odstavce
\usepackage{mathtools}
\usepackage{amsfonts}
\usepackage{amsmath}
\usepackage{amssymb}
\usepackage{graphicx}
\usepackage{enumitem}
\usepackage{subfig}
\usepackage{float}
\usepackage[czech]{babel}
\usepackage{mathdots}
\usepackage{slashbox}

\begin{document}
\begin{center}
\large NMMB538 - DÚ7

\normalsize Jan Oupický
\end{center}
\vspace{1\baselineskip}

\section{}
Z definice $[2]$ víme, že $\forall \alpha \in D: [2](\alpha) = \alpha \oplus \alpha$. Vyjádříme tedy vzorec pro výpočet bodu $\alpha \oplus \alpha \in D$ křivky $V_w$.

Nejprve vyjádříme racionální funkci. Použijeme vzorec pro součet bodů a to, že pro body $(\alpha_1, \alpha_2) \in D$ platí $\alpha_2^2 = \alpha_1^3+a \alpha_1^2 + b \alpha$. $\gamma$ bude značit reprezentanta rac. zobrazení $\gamma(\alpha) = [2](\alpha), \gamma = (\gamma_1, \gamma_2), \gamma_i \in K(x_1, x_2)$. 

Vyjádříme nejprve $\gamma_1(x_1,x_2)$. Ze vzorců pro součet stejného bodu vyjde, že:
\begin{gather*}
\gamma_1(x_1,x_2) = \frac{-8x_1x_2^2 - 4ax_2^2 + 9x_1^4+12ax_1^3+6bx_1^2+4a^2x_1^2+4abx_1+b^2}{4x_2^2}\\
\text{zasubstituujeme za $x_2^2$ v čitateli a vyjde:}\\
\gamma_1(x_1,x_2) = \frac{x_1^4-2bx_1^2+b^2}{4x_2^2} = \left( \frac{x_1^2-b}{2x_2} \right)^2
\end{gather*}
Nyní spočteme $\gamma_2(x_1,x_2)$:
\begin{gather*}
\gamma_2(x_1,x_2) = \frac{3x_1^3+2ax_1^2+bx_1}{2x_2}-\frac{3x_1^3+2ax_1+b}{2x_2}\cdot \frac{x_1^4-2bx_1^2+b^2}{4x_2^2}-x_2\\
\text{Převedení na společného jmenovatele a použití substituce za $x_2^2$ v čitateli:}\\
\gamma_2(x_1,x_2) = \frac{x_1^6+2ax_1^5+5bx_1^4-5b^2x_1^2-2ab^2x_1-b^3}{8x_2^3}
\end{gather*}
Máme tedy reprezentanty $K-$racionálního zobrazení $[2]: D \rightarrow D$, kde \\$[2] = (\gamma_1+(w), \gamma_2+(w))$. Sestrojíme nyní projektivní reprezentaci zobrazení $[2]$ jako v předchozím úkolu pomocí lemma M.3.
\begin{gather*}
\gamma_i = \frac{a_i}{b_i}, a_i, b_i \in K[x_1,x_2], b_i \neq 0\\
A'_1 = \widehat{a_1} X_3^2 = (X_1^4-2bX_1^2X_3^2 + b^2X_3^4)X_3^2\\
A'_2 = \widehat{a_2} X_3^3 = (X_1^6+2aX_1^5X_3+5bX_1^4X_3^2-5b^2X_1^2X_3^4-2ab^2X_1X_3^5-b^3X_3^6) X_3^3\\
B'_1 = 4X_2^2X_3^4\\
B'_2 = 8X_2^3X_3^6\\
\implies\\
(A_1 : A_2 : A_3) = (A'_1B'_2 : A'_2B'_1 : B'_1B'_2)\\
\text{dosazení a zkrácení:}\\
(A_1 : A_2 : A_3) = (2(X_1^4-2bX_1^2X_3^2 + b^2X_3^4)X_2X_3 : \widehat{a_2} : 4X_2^3X_3^3)
\end{gather*}
\section{}
Pro reprezentaci $[2] = (A_1 : A_2 : A_3)$ výše platí $\deg_{X_2}(A_1) = 1, \deg_{X_2}(A_2) = 0$. V posledním členu $A_3$ nahradíme $X_2^2X_3$ za $X_1^3+aX_1^2X_3+bX_1X_3^2$. Máme tedy novou reprezentaci, která splňuje podmínky:
\begin{gather*}
(A'_1 : A'_2 : A'_3) = (2(X_1^4-2bX_1^2X_3^2 + b^2X_3^4)X_2X_3 : \widehat{a_2} : 4X_2X_3^2(X_1^3+aX_1^2X_3+bX_1X_3^2))
\end{gather*}

Pro reprezentaci $[2] = (B_1 : B_2 : B_3), \deg_{X_1}(B_i)\leq 2$ využijeme substituci $X_1^3 = X_2^2X_3-aX_1^2X_3-bX_1X_3^2$. Každá substituce sníží stupeň (v $X_1$) polynomu o 1. Zřejmě se dostaneme do tvaru, kde $\deg_{X_1}(B_i)\leq 2$. Substituce do 1. členu:
\begin{gather*}
B_1 = 2X_2X_3^2((a^2-3b)X_1^2X_3+X_1X_2^2+abX_1X_3^2-aX_2^2X_3+b^2X_3^3)\\
\end{gather*}
$B_2$ získáme opakovanou substitucí za $X_1^3$. Polynom je tvaru:
\begin{gather*}
B_2 = -X_3^2(-X_2^4+X_2^2X_3f_1(X_1,X_2,X_3) + X_3^2f_2(X_1,X_2,X_3)), \deg_{X_1}(f_i) = i
\end{gather*}

\section{}
Bod $\infty = (0:1:0) \in D$. Zřejmě $A'_1,A'_3(\infty) = 0$ (násobky $X_3$) a $A'_2(\infty) = 0$ (jsou tam jen monočleny $X_1^iX_3^j$). Obdobně $B_i(\infty)=0$ (všechno to jsou násobky $X_3$).

Pokud ale definujeme $B'_i = \frac{B_i}{X_3^2}$. Poté $B'_2(\infty) = 1$.

\section{}

\end{document}

