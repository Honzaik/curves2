\documentclass[12pt, a4paper]{article}
\usepackage[margin=1in]{geometry}
\usepackage[utf8x]{inputenc}
\usepackage{indentfirst} %indentace prvního odstavce
\usepackage{mathtools}
\usepackage{amsfonts}
\usepackage{amsmath}
\usepackage{amssymb}
\usepackage{graphicx}
\usepackage{enumitem}
\usepackage{subfig}
\usepackage{float}
\usepackage[czech]{babel}
\usepackage{mathdots}
\usepackage{slashbox}

\begin{document}
\begin{center}
\large NMMB538 - DÚ7

\normalsize Jan Oupický
\end{center}
\vspace{1\baselineskip}

\section{}
Z definice $[2]$ víme, že $\forall \alpha \in D: [2](\alpha) = \alpha \oplus \alpha$. Vyjádříme tedy vzorec pro výpočet bodu $\alpha \oplus \alpha \in D$ křivky $V_w$.

Nejprve vyjádříme racionální funkci. Použijeme vzorec pro součet bodů a to, že pro body $(\alpha_1, \alpha_2) \in D$ platí $\alpha_2^2 = \alpha_1^3+a \alpha_1^2 + b \alpha$. $\gamma$ bude značit reprezentanta rac. zobrazení $\gamma(\alpha) = [2](\alpha), \gamma = (\gamma_1, \gamma_2), \gamma_i \in K(x_1, x_2)$. 

Vyjádříme nejprve $\gamma_1(x_1,x_2)$. Ze vzorců pro součet stejného bodu vyjde, že:
\begin{gather*}
\gamma_1(x_1,x_2) = \frac{-8x_1x_2^2 - 4ax_2^2 + 9x_1^4+12ax_1^3+6bx_1^2+4a^2x_1^2+4abx_1+b^2}{4x_2^2}\\
\text{zasubstituujeme za $x_2^2$ v čitateli a vyjde:}\\
\gamma_1(x_1,x_2) = \frac{x_1^4-2bx_1^2+b^2}{4x_2^2} = \left( \frac{x_1^2-b}{2x_2} \right)^2
\end{gather*}
Nyní spočteme $\gamma_2(x_1,x_2)$:
\begin{gather*}
\gamma_2(x_1,x_2) = \frac{3x_1^3+2ax_1^2+bx_1}{2x_2}-\frac{3x_1^2+2ax_1+b}{2x_2}\cdot \frac{x_1^4-2bx_1^2+b^2}{4x_2^2}-x_2\\
\text{Převedení na společného jmenovatele a použití substituce za $x_2^2$ v čitateli:}\\
\gamma_2(x_1,x_2) = \frac{x_1^6+2ax_1^5+5bx_1^4-5b^2x_1^2-2ab^2x_1-b^3}{8x_2^3}
\end{gather*}
Máme tedy reprezentanty $K-$racionálního zobrazení $[2]: D \rightarrow D$, kde \\$[2] = (\gamma_1+(w), \gamma_2+(w))$. Sestrojíme nyní projektivní reprezentaci zobrazení $[2]$ jako v předchozím úkolu pomocí lemma M.3.
\begin{gather*}
\gamma_i = \frac{a_i}{b_i}, a_i, b_i \in K[x_1,x_2], b_i \neq 0\\
A'_1 = \widehat{a_1} X_3^2 = (X_1^4-2bX_1^2X_3^2 + b^2X_3^4)X_3^2\\
A'_2 = \widehat{a_2} X_3^3 = (X_1^6+2aX_1^5X_3+5bX_1^4X_3^2-5b^2X_1^2X_3^4-2ab^2X_1X_3^5-b^3X_3^6) X_3^3\\
B'_1 = 4X_2^2X_3^4\\
B'_2 = 8X_2^3X_3^6\\
\implies\\
(A_1 : A_2 : A_3) = (A'_1B'_2 : A'_2B'_1 : B'_1B'_2)\\
\text{dosazení a zkrácení:}\\
(A_1 : A_2 : A_3) = (2(X_1^4-2bX_1^2X_3^2 + b^2X_3^4)X_2X_3 : \widehat{a_2} : 8X_2^3X_3^3)
\end{gather*}
\section{}
Pro reprezentaci $[2] = (A_1 : A_2 : A_3)$ výše platí $\deg_{X_2}(A_1) = 1, \deg_{X_2}(A_2) = 0$. V posledním členu $A_3$ nahradíme $X_2^2X_3$ za $X_1^3+aX_1^2X_3+bX_1X_3^2$. Máme tedy novou reprezentaci, která splňuje podmínky:
\begin{gather*}
(A'_1 : A'_2 : A'_3) = (2(X_1^4-2bX_1^2X_3^2 + b^2X_3^4)X_2X_3 : \widehat{a_2} : 8X_2X_3^2(X_1^3+aX_1^2X_3+bX_1X_3^2))
\end{gather*}

Pro reprezentaci $[2] = (B_1 : B_2 : B_3), \deg_{X_1}(B_i)\leq 2$ využijeme substituci $X_1^3 = X_2^2X_3-aX_1^2X_3-bX_1X_3^2$. Každá substituce sníží stupeň (v $X_1$) polynomu o 1. Zřejmě se dostaneme do tvaru, kde $\deg_{X_1}(B_i)\leq 2$. Substituce do 1. členu:
\begin{gather*}
B_1 = 2X_2X_3^2((a^2-3b)X_1^2X_3+X_1X_2^2+abX_1X_3^2-aX_2^2X_3+b^2X_3^3)\\
\end{gather*}
$B_2$ získáme opakovanou substitucí za $X_1^3$. Polynom je tvaru:
\begin{gather*}
B_2 = -X_3^2(-X_2^4+X_2^2X_3f_1(X_1,X_2,X_3) + X_3^2f_2(X_1,X_2,X_3)), \deg_{X_1}(f_i) = i \text{ kde }\\
f_1(X_1,X_2,X_3) = a^2 X_1 +a^3 (-X_3)+5 a b X_3-3 b X_1\\
f_2(X_1,X_2,X_3) = -6 a^2 b X_1^2+a^3 b X_1 X_3+a^4 X_1^2-3 a b^2 X_1 X_3+b^2 \left(b X_3^2+9 X_1^2\right)
\end{gather*}

\section{}
Bod $\infty = (0:1:0) \in D$. Zřejmě $A'_1,A'_3(\infty) = 0$ (násobky $X_3$) a $A'_2(\infty) = 0$ (jsou tam jen monočleny $X_1^iX_3^j$). Obdobně $B_i(\infty)=0$ (všechno to jsou násobky $X_3$).

Pokud ale definujeme $B'_i = \frac{B_i}{X_3^2}$. Poté $B'_2(\infty) = 1$.

Pro zbylé body nyní můžeme uvažovat $X_3 = 1$. Pokud 2. souřadnice bodu z $D$ není 0, tak $A'_3(\alpha) \neq 0$ pro každý takový bod $\alpha \in D$ (protože $X_2 \neq 0$ a $X_3 = 1$).

Zbývají 3 body $D$ tvaru $\alpha = (\alpha_1, 0) \in D$. Možná $\alpha_1$ jsou $\{0, r_1, r_2\}$, kde $r_i, i=1,2$ jsou kořeny $x^2+ax+b$. Jelikož $X_2=0$, tak jsou relevantní jen polynomy $A'_2$ a $B_2$ (ostatní jsou $0$).

Pro bod $(0,0)$ se to redukuje na zda $f_2(0,0,1) = 0$? Po dosazení platí $f_2(0,0,1) = b^3$ neboli $B'_2(\alpha) \neq 0$.

Polynom $f_2(x,0,1) = x^2(a^4-6a^2+9b^2)+x(a^3b-3ab^2)+b^3$ . Tento polynom nemá žádné společné kořeny s polynomem $x^2+ax+b$. Takže $f_2(r_i,0,1) \neq 0, i = 1,2 \implies B'_2((r_i:0:1)) \neq 0, i = 1,2$.

\section{}
Označme
\begin{gather*}
\rho_1 = \left( \frac{x_1^2-b}{2x_2} \right)^2\\
\rho_2 = \frac{x_1^6+2ax_1^5+5bx_1^4-5b^2x_1^2-2ab^2x_1-b^3}{8x_2^3}
\end{gather*}

Všimneme si, že platí:
\begin{gather*}
\rho_1 = \frac{v^2}{4u^2} = \left( \frac{v}{2u} \right)^2
\end{gather*}
Kde $u = t^2, v = st, t = \frac{x_2}{x_1}, s = \frac{b-x_1^2}{x_1}$ z minulých úloh. Poněkud nepřesně zde ztotožňujeme $x_1 + (w) = x_1, x_2= x_2 + (w)$. Obdobně také $\rho_2$ jde vyjádřit pomocí $u,v$. 
\begin{gather*}
\rho_2 = \rho_2  \frac{x_1^2}{x_1^2} = \frac{(b-x_1^2)(-1)(x_1^6+2ax_1^5+6bx_1^4+2abx_1^3+b^2x_1^2)}{8x_1^2x_2^3} = \\
\frac{v}{8} \frac{-(x_1^6+2ax_1^5+6bx_1^4+2abx_1^3+b^2x_1^2)}{x_2^3} \frac{x_2}{x_2} = \frac{v}{8} \frac{-((x_1^3+ax_1^2+bx_1)^2-a^2x_1^4+4bx_1^4)}{x_2^4} =\\
\frac{v}{8} \frac{(-x_2^4+a^2x_1^4-4bx_1^4)}{x_2^4} \frac{x_1^4}{x_1^4} = \frac{v}{8} \frac{a^2-4b-\frac{x_2^4}{x_1^4}}{\frac{x_2^4}{x_1^4}} = \frac{v(a^2-4b-u^2)}{8u^2}
\end{gather*}
Neboli $\text{Im}([2]^*) = K(\frac{v^2}{4u^2},\frac{v(a^2-4b-u^2)}{u^2})$. Ukažeme, že $[K(u,v):\text{Im}([2]^*)]=2$. 

Zvolme $d = \sqrt{\frac{v^2}{u^2}} = \frac{v}{u}$. Ukážeme, že $K(\rho_1,\rho_2,d)=K(u,v)$. Poté zřejmě $min_{d,K(\rho_1,\rho_2)}(T)=T^2-4\rho_1$, což je ireducibilní a separabilní polynom nad $K(\rho_1, \rho_2)$ (kořeny $\pm d$).

\begin{gather*}
\rho_2 d^{-1} = \frac{v(a^2-4b-u^2)}{u^2} \frac{u}{v} = \frac{a^2u-4bu-u^3}{u^2} = \frac{a^2u-4bu-(v^2+2au^2-a^2u+4bu)}{u^2} =\\
\frac{-v^2-2au^2+2a^2u-8bu}{u^2} = -\frac{v^2}{u^2} -2a + \frac{2a^2-8b}{u} \implies\\
\rho_2 d^{-1} + d^2 +2a = \frac{2a^2-8b}{u} \implies u = \frac{2a^2-8b}{\rho_2 d^{-1} + d^2 +2a} \in K(\rho_1, \rho_2, d)\\
d = \frac{v}{u} \implies d u = v \implies v \in K(\rho_1, \rho_2, d)
\end{gather*}
V úpravách jsme použili rovnost $v^2 = u^3-2au^2+u(a^2-4b)$, kterou jsme dokázali v úkolu 4.

Tedy $K(u,v)/\text{Im}([2]^*)$ je separabilní a stupně 2. Z minulých úkolů víme, že $K(D)/K(u,v)$ je galoisovo a stupně 2. Z toho vyplývá, že $K(D)/\text{Im}([2]^*)$ je stupně $2\cdot2$ a je separabilní. Ekvivalentně $[2]$ je separabilní isogeny a $\deg([2])=4$.

V minulých úlohách jsme dokázali, že $D$ je smooth, tedy dle X.13 $\text{Im}([2])=D$.

Z definice isogeny víme, že $[2](\infty) = \infty$. Pro zjištění jádra chceme vědět, jaké body $\alpha \in D$ se zobrazí $[2]$ na $\infty$ neboli pro jaké body platí $\rho_1 = \frac{a_1}{b_1}, \rho_2 = \frac{a_2}{b_2}$ platí $b_i(\alpha)=0$. Vidíme, že musí $\alpha_2 = 0$. $D$ je smooth, tedy máme $3$ různé body $P_0,P_1,P_2 = (0,0),(r_1,0),(r_2,0) \in D$, kde 2. souřadnice je 0.

$K(D)/\text{Im}([2]^*)$ je separabilní, tedy dle T.15 $[K(D)/\text{Im}([2]^*)]_s = [K(D)/\text{Im}([2]^*)] = 4$, tedy $|\text{Ker}([2])|=4$. Našli jsme 4 různé body a tedy $\text{Ker}([2]) = \{\infty, P_0, P_1, P_2\}$. 

Přesně jsme neukázali, že $d \notin K(\rho_1, \rho_2)$. Pokud by to platilo, tak $\deg([2]) = 2$ a $[2]$ separabilní. Nalezli jsme ale 4 různé body, které jsou prvky $\text{Ker}([2])$, tedy spor s T.15.

\section{}
Zadání splňuje předpoklady T.17, tedy $\text{Gal}(K(D)|\text{Im}([2]^*)) = \{t^*_\alpha| \alpha \in \{\infty, P_1, P_2, P_3\}\}$.
Z definice $t_\infty (\alpha) = \infty \oplus \alpha = \alpha$, neboli $t_\infty = [1] \iff t_\infty^* = (x,y), (X:Y:Z)$.

Dále $P_1 = (0,0) \implies t_{(0,0)}(\alpha) = (0,0) \oplus \alpha$.

Spočteme tedy afinní reprezentanty $t_{(0,0)}^*$, které jsou $t_{(0,0)}^* = \left( \frac{-b}{x_1}, \frac{-bx_2}{x_1^2} \right)$. Z toho dostaneme klasickým způsobem projektivní reprezentanty: 
\[(-bX_1X_3 : -bX_2X_3 : X_1^2) \]

Zbylé 2 translace korespondují s body $P_2,P_3$, kde $P_2 = (r_1, 0), P_3 = (r_2, 0)$ pro $r_1,r_2$ platí $r_i^2+ar_i+b = 0$ (jsou to kořeny $x_1^2+ax_1+b$). Opět z definice $t_{(r_1,0)}(\alpha) = (r_1,0) \oplus \alpha$. Pak jsou afinní reprezentanti $t_{(r_1,0)}^*$: 
\begin{gather*}
\left(\frac{r_1(a+r_1+x_1)}{x_1-r_1}, \frac{x_2(ar_1+2b)}{(x_1-r_1)^2} \right )\\
\text{a projektivní:}\\
\left(r_1((a+r_1)X_3 + X_1)(X_1-r_1X_3)X_3 : X_1^2X_2-2r_1X_1X_2X_3+bX_2X_3^2 : (X_1-r_1X_3)^2X_3 \right)
\end{gather*}

K 1. affinímu reprezentantovi jsme se dostali následnovně:
\begin{gather*}
kek
\end{gather*}

A k druhému:
\begin{gather*}
\gamma_2(x_1,x_2) = \frac{x_2}{x_1-r_1} \cdot
\end{gather*}

Pro $P_3 = (r_2,0)$, $t_{(r_2,0)}(\alpha) = (r_2, 0) \oplus \alpha$ můžeme provést stejný postup. Reprezentanti $t_{(r_2,0)}^*$ jsou tedy:
\begin{gather*}
\left(\frac{r_2(a+r_2+x_1)}{x_1-r_2}, \frac{x_1^2x_2-2r_2x_1x_2 + bx_2}{(x_1-r_2)^2} \right )\\
\text{a projektivní:}\\
\left(r_2((a+r_2)X_3 + X_1)(X_1-r_2X_3)X_3 : X_1^2X_2-2r_2X_1X_2X_3+bX_2X_3^2 : (X_1-r_2X_3)^2X_3 \right)
\end{gather*}
\end{document}