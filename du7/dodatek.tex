\documentclass[12pt, a4paper]{article}
\usepackage[margin=1in]{geometry}
\usepackage[utf8x]{inputenc}
\usepackage{indentfirst} %indentace prvního odstavce
\usepackage{mathtools}
\usepackage{amsfonts}
\usepackage{amsmath}
\usepackage{amssymb}
\usepackage{graphicx}
\usepackage{enumitem}
\usepackage{subfig}
\usepackage{float}
\usepackage[czech]{babel}
\usepackage{mathdots}
\usepackage{slashbox}

\begin{document}

nechť $f(x_1,x_2)$ značí 1. reprezentanta
\begin{gather*}
r^2 + ar+b = 0\\
\lambda = \frac{x_2}{x_1-r} \implies f(x_1,x_2) = - r - x_1 + \frac{x_2^2}{(x_1-r)^2} - a \implies\\
f(x_1,x_2) = \frac{(-r-x_1)(x_1-r)^2+x_2^2-a(x_1-r)^2}{(x_1-r)^2} \text{ roznásobení a sub. za $x_2^2$} =\\
\frac{x_1^2r+x_1r^2-r^3+2x_1ar-ar^2+bx_1}{(x_1-r)^2} = \frac{-ar^2-r^3+arx_1+x_1(b+ar+r^2)+rx_1^2}{(x_1-r)^2} \implies\\
b+ar+r^2 = 0 \implies\\
\frac{-ar^2-r^3+arx_1+rx_1^2}{(x_1-r)^2} = \frac{r(x_1-r)(a+r+x_1)}{(x_1-r)^2} = \frac{r(a+r+x_1)}{x_1-r}
\end{gather*}
\end{document}